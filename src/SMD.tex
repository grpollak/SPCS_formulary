\subsection{SMD/Langerin-Dynamik}
\label{subsec:SMD/LangerinDynamik}
\begin{sectionbox}[Einführung]\nospacing
  Ist eine stochastische Variante der MD zur Beschreibung komplexer Systeme.\\
  \imp{Aufgabenstellung}: simulation von ``Teilchen'' die einer schneller fluktuierenden Umgebung ausgestzt sind Bsp.:
  \begin{numberlist}
    \item Fettblasen in Suppe.
    \item Polymere in Lösung.
    \item Pollen in einem Wassertropfen.
    \item   Gefragt sind nicht die exakten Werte von $\vvec(t)$ und $\rvec(t)$ sondern deren Wahrscheinlichkeitsverteilung.
  \end{numberlist}
  Die (Lösungsmittel-)Teilchen verursachen rasch variierende stochastische Kräfte und Dämpfung.
  \imp{Idee}: Wenn die explizite Dynamik dieser Umgebung nicht von Interesse ist, dann ist die Dynamik des Systems mittels zusätzlichen
    Reibungsterms und einer flukturierenden Stochastischen Kraft $\Fvecst$ modelierbar. 
\end{sectionbox}
\begin{notebox}[Nebenbemerkung]
  Ist ein Spezialfall der Brownschen Dynamik (Brownian MD).
\end{notebox}
\begin{lawbox}\nospacing
  \begin{law}[Langerin-Geleichung \tc{black}{(pro. Teilchen)}]
    \begin{align}
      &m\dot{\vvec}+\underbrace{\rd{m\gammac\vvec}}_{\mathclap{\text{ratio}}}=\Fvec+\underbrace{\rd{\Fvecst}}_{\mathclap{\text{actio}}}\nalign
      &\gammac:\quad \text{\rdb{Atomarer Reibungskoeffizient}}\label{eq:atomarerReibungskoeffizient}
    \end{align}
  \end{law}
\end{lawbox}
\begin{notebox}[Bemerkungen]
  \begin{numberlist}
      \item Die Langevin Gleichung ist eine stochastische Differentialgleichung.
      \item Da $\Fvec(t)$ eine Zufallsvariabel ist, muss auch $\vvec(t)$ eine Zufallsvariabel sein.
  \end{numberlist}
\end{notebox}
\begin{sectionbox}[Formale analytische Lösung für $F\equiv0$]\nospacing
  \begin{align}
    \Fvec=0&&\Longleftrightarrow&&\Wpot=0&&\Longleftrightarrow&&\text{freies Teilchen}
  \end{align}
  \begin{align*}
    &\gammac&\text{Materialkonstante/Reibungskoeffizient}
  \end{align*}
  \begin{align}
    &\imp{Ansatz}&\ul{\vvec}(t)=\vvec_{0}\cdot\e^{-\gammac t}+\frac{\e^{-\gammac t}}{m}\int_{0}^{t}\e^{\gammac t'}\Fvecst(t)\diff t'\label{eq:AnsatzWO}
  \end{align}
\end{sectionbox}
\begin{proofbox}\nospacing
   \begin{proof}
      \begin{align*}
        \ul[ulc2]{\frac{\diff\ul{\vvec(t)}}{\diff t}\eqs{\rd{P.R.}}}&-\ul{\gammac\left[\vvec_{0}\e^{-\gammac t}+\frac{\e^{-\gammac t}}{m}\int_{0}^{t}\e^{-\gammac t'}\Fvecst(t')\diff t'\right]}\nalign
                                                    &+\frac{\e^{-\gammac t}}{m}\e^{\gammac t}\Fvecst(t)=-\gammac\vvec(t)+\frac{\Fvecst(t)}{m}
      \end{align*}
      \begin{align*}
        &\Rightarrow&m\ul[ulc2]{\difrac{\vvec(t)}{t}}+m\gammac\vvec(t)=\Fvecst
      \end{align*}
   \end{proof} 
\end{proofbox}
\subsection{Mittlere Grössen}
\begin{sectionbox}\nospacing
  Sind besonders wichtig/nützlich für die Langerindynamik, da $\Fvecst$ eine Stochstische Kraft ist.
  \begin{flalign*}
    &\text{\imp{Recall} :}\cref{eq:ergodenHypothese}&\underbrace{\overline{{\vvec}}(t)}_{\text{Zeitmittel}}=\lim_{t\to\infty}\underbrace{\obs{\vvec(t)}}_{\text{Scharmittel}}
  \end{flalign*}
  \imp{Annahmen}:
  \begin{circlelist}
      \item Die Fluktuationen zeigen keine Tendenz und geben im Mittel 0.
        \begin{align}
          \boxed{\obs{\Fvecst(t)}=0}
        \end{align}
          \item Es gibt keine Korrelation zwischen Stössen der Teilchen d.h. Stösse der Teilchen zwischen $t_1$ und $t_2$ sind unkorreliert
        und werden als unabhängige Zufallsvariablen betrachtet.
        \begin{align}
          \text{\rdb{Autokorrelationsfunktion/Gleitender Mittelwert}}\nonumber\nalign
          \boxed{\obs{\Fvecst(t_1)\cdot\Fvecst(t_2)}\propto\delta(t_1-t_2):=\begin{cases}
              1 & \text{für }t_1=t_2\\
              0 & \text{für }t_1\neq t_2
            \end{cases}}
        \end{align}
  \end{circlelist}
\end{sectionbox}
\begin{sectionbox}[\subsubsection{Mittlere Geschwindigkeit für ein Teilchen}]\nospacing
  \begin{align*}
    \obs{\ul{\vvec(t)}}=&\obs{\vvec_0\e^{-\gammac t}+\frac{\e^{-\gammac t}}{m}\int_{0}^t\e^{\gammac t'}\Fvecst(t')\diff t'}\nalign
                          =&\vvec_{0}\e^{-\gammac t}+\frac{\e^{-\gammac t}}{m}\int_{0}^t\e^{\gammac t'}\underbrace{\obs{\Fvecst(t')}}_{\eqs{\scalebox{0.7}{\circledItem{1}}}0}\diff t'
  \end{align*}
  \begin{align}
    &\Rightarrow&\obs{\vvec(t)}=\vvec_0\e^{-\gammac t}&&\Rightarrow&&\boxed{\bar{\vvec}(t)=\lim_{t\to\infty}\left(\vvec_0\e^{-\gammac t}\right)=0}
  \end{align}
\end{sectionbox}
\begin{notebox}[Bemerkung]
  Die mittlere Geschwindikeit verschwindet also, dies macht Sinn, da $\Fvec=-\nabla\Wpot=0$.
  $\Longleftrightarrow$ nur $\Fvecst$ bewegt das Teilchen und diese Kraft verschwindet im Mittel $\Rightarrow$ auch $\vvec^{\text{st}}$
  sollte im mittel Verschwinden.
\end{notebox}
\begin{theorembox}
  \begin{theorem}[Satz von Fubini]\nospacing
    \todo[inline]{Add}
   \end{theorem} 
\end{theorembox}
\begin{corbox}\nospacing
  \begin{cor}[Quadrat von Integrallen]
    \begin{equation}\label{eq:QuadratVonIntegralen}
      \left(\int\limits_{\mca}^{\mcb}f(\vec{x})\diff \vec{x}\right)^2=\int\limits_{\mca}^{\mcb}\int\limits_{\mca}^{\mcb}f(\vec{x})f(\vec{y})\diff \vec{x}\diff\vec{y}
    \end{equation}
  \end{cor}  
  \todo[inline]{Add proof}
\end{corbox}
\begin{sectionbox}[\subsubsection{Mittlere kinetische Energie \tc{black}{$\Ez_{\text{kin}}\eqs{}\frac{1}{2}m\vvec^2$}}]\nospacing
  Um die kinetische Energie zu berechnen benötigen wir $\vvec(t)^2$.
  \begin{align*}
        \obs{\ul{\vvec(t)}^2}=&\obs{\left(\vvec_0\e^{-\gammac t}+\frac{\e^{-\gammac t}}{m}\int_{0}^t\e^{\gammac t'}\Fvecst(t')\diff t'\right)^2}\nalign
                                =&\vvec_{0}^2\e^{-2\gammac t}+2\vvec_0\frac{\e^{-2\gammac t}}{m}\int_{0}^t\e^{\gammac
                             t'}\underbrace{\obs{\Fvecst(t')}}_{\eqs{\scalebox{0.7}{\circledItem{1}}}0}\diff t'\nalign
    &+\left(\frac{\e^{-\gammac t}}{m}\int_{0}^t\e^{\gammac t'}\Fvecst(t')\diff t'\right)^2 \eqs{\text{\rd{Fub.Thr.}}}\vvec_{0}^2\e^{-2\gammac t}\nalign
    &+\frac{\e^{-\gammac t}}{m^2}\int\limits_{0}^t\int\limits_{0}^t\e^{\gammac (t'+t'')}\underbrace{\obs{\Fvecst(t')\cdot\Fvecst(t'')}}_{\eqs{\scalebox{0.7}{\circledItem{2}}}:\mcc\delta(t'-t')}\diff t'\diff t''\nalign
      \eqs{t'=t''}&\vvec_{0}^2\e^{-2\gammac t}+\frac{\e^{-\gammac t}}{m^2}\mcc\int\limits_{0}^t\e^{2\gammac t'}\diff t'\nalign
                    =&\vvec_{0}^2\e^{-2\gammac t}+\frac{\mcc\e^{-\gammac t}}{m^2}\left.\left[\frac{1}{2\gammac}\e^{\gammac 2t'}\right]\right\rvert_{0}^t\nalign
                       =&\ul[ulc2]{\vvec_{0}^2\e^{-2\gammac t}-\frac{\mcc\e^{-2\gammac t}}{2\gammac m^2}+\frac{\mcc}{2\gammac m^2}}
  \end{align*}
  Damit folgt:
  \begin{align*}
    \overline{\vvec(t)^2}=\lim_{t\to\infty}\obs{\ul[ulc2]{\vvec(t)^2}}=\frac{\mcc}{2\gammac m^2}
  \end{align*}
\end{sectionbox}
\begin{theorembox}\nospacing
   \begin{theorem}[Äquipartitionstheorem]
     Im thermischen Gleichgewicht bei der Temperatur $\Tz$ besitzt jeder Freiheitsgrad $f$ die gleiche mittlere Energie $\obs{\Ez}=\frac{1}{2}\kb T$:
     \begin{align}
       \obs{\Ez}=\frac{f}{2}\kb\Tz&f: \text{ Anzahl an Freiheitsgraden}\label{eq:Aquipartitionstheorem}
     \end{align}
   \end{theorem} 
\end{theorembox}
\begin{notebox}[Bemerkung]\nospacing
  Ein punktförmiges Teilchen hat drei\\ Translationsfreiheitsgrade
  \begin{align}
      \obs{\Ez}=\frac{3}{2}\kb\Tz\label{eq:GasFreiheitsgrade}
  \end{align}
\end{notebox}
\begin{sectionbox}[Bestimmung von $\mcc$]\nospacing
  Mit dem Äquipartitionstheorem \cref{eq:Aquipartitionstheorem} und dem Fakt das die Energie eins Gases durch die kinetische Energie der
  Atome gegeben ist $\Ez_{kin}=\frac{1}{2}m\vvec^2$ (pro. Teilchen), folgt:
  \begin{align*}
    \overline{\vvec(t)^2}=\lim_{t\to\infty}\obs{\ul[ulc2]{\vvec(t)^2}}=\frac{\mcc}{2\gammac
    m^2}\eqs{!}\frac{2\overline{\Ez_{\text{kin}}}}{m}\eqs{\cref{eq:Aquipartitionstheorem}}\frac{3\kb\Tz}{m}
  \end{align*}
  \begin{align*}
    \Rightarrow \mcc=6\gammac m\kb\Tz
  \end{align*}
\end{sectionbox}
\begin{defnbox}
  \begin{defn}[Autokorrelationsfunktion]
    \begin{equation}
      \obs{\Fvecst(t_1)\cdot\Fvecst(t_2)}=\mcc[6\gammac m\kb\Tz]\delta(t'-t'')\label{eq:autokorrelationsfunktion}
    \end{equation}
  \end{defn}
\end{defnbox}
\begin{theorembox}\nospacing
    \begin{theorem}[\\Second Fundamental Theorem of Calculus]
      \begin{equation}
        \difrac{}{x}\int_{\mca}^{x}f(t)\diff t=f(x)\label{eq:SFTC}
      \end{equation}
    \end{theorem}
\end{theorembox}
\begin{notebox}[Einleitung:\\ Diffusion und mittleres Verschiebungsquadrat]
  Unter Diffusion versteht man die Durchmischung von zwei oder mehreren verschiedenen, miteinander in Berührung stehenden Stoffen.\\
  Diese Durchmischung entsteht durch die thermische Bewegung (Brown'sche Molekularbewegung) der Teilchen und verläuft gleichmäßig in alle
  Richtungen.\\
  Die treibende Kraft der Diffusion ist der lokale Konzentrationsunterschied der diffundierenden Teilchen. Die Diffusion führt ohne
  Einwirkung von äußeren Kräften zum Abbau des Konzentrationsgradienten.\\
  Die Geschwindigkeit der Diffusion wird über das mittlere Verschiebungsquadrat $\obs{\rvec^2}$ beschrieben.
  \end{notebox}
\begin{sectionbox}[\subsubsection{Das Mittlere Verschiebungsquadrat}]\nospacing
  Ist ein Maß für die Strecke, die ein Teilchen im Mittel, von einem gewissen Refernzpunkt $\rvec(t_0)=\rvec_{0}$ in einer gewissen Zeit zurücklegt.
  Das mitteler Auslenkungs/Verschiebungsquadrat ist definiert durch:
  \begin{align*}
    &\obs{\abs{\rvec(t)-\rvec_0}^2}\eqs{\rvec_0=0}\obs{\rvec(t)^2}\text{   mit   }\rvec(t)=\rvec_0+\int_{0}^t\vvec(\mca[t'])\diff t'
  \end{align*}
  \begin{align*}
    \rvec(t)-&\rvec_0=\int_{0}^t\ul{\vvec(t')}\diff t'\nalign
                       \eqs{\cref{eq:AnsatzWO}}&\int_{0}^t\left[\vvec_{0}\cdot\e^{-\gammac t'}+\frac{\e^{-\gammac t'}}{m}\int_{0}^{t'}\e^{\gammac t''}\Fvecst(t'')\diff t''\right]\diff t'
  \end{align*}
  \begin{align*}
    \hspace{2em}=&\vvec_0 \left.\left[-\frac{1}{\gammac}\e^{-\gammac t'}\right]\right\rvert_0^t+\rd{\int\limits_{0}^{t}}\underbrace{\frac{\e^{-\gammac \rd{t'}}}{m}}_{\mca[u']}
                   \underbrace{\int_{0}^{\rd{t'}}\e^{\gammac t''}\Fvecst\diff t''}_{\mcb[v]}\rd{\diff}\rd{t'}\nalign
    \eqs{\text{\rd{P.I.}}}&\frac{\vvec_0}{\gammac}\left(1-\e^{-\gammac t}\right)+
                                           \frac{1}{m}\left.\left[\underbrace{-\frac{\e^{-\gammac \rd{t'}}}{\gammac}}_{\mca[u]}\underbrace{\int_{0}^{\rd{t'}}\e^{\gammac t''}\Fvecst\diff t''}_{\mcb[v]}\right]\right\rvert_{\tc{green}{0}}^{\tc{green}{t}}\nalign
    &\underbrace{-\frac{1}{m}\rd{\int\limits_0^{t}}\left[\underbrace{-\frac{\e^{-\gammac \rd{t'}}}{\gammac}}_{\mca[u]}
      \cdot\underbrace{\frac{\diff}{\diff \rd{t'}}\int_{0}^{\rd{t'}}\e^{\gammac t''}\Fvecst\diff t''}_{\mcb[v']}\right]\diff\rd{t'}}_{
      \eqs{\cref{eq:SFTC}}-\frac{1}{\gammac m}\rd{\int\limits_0^{t}}\underbrace{-\e^{-\gammac \rd{t'}}}_{\mca[u]}\underbrace{\e^{\gammac\rd{t'}}\Fvecst}_{\mcb[v']}\diff\rd{t'}
      =\frac{1}{\gammac m}\rd{\int\limits_0^{t}}\Fvecst\diff\rd{t'}}\nalign
      =\frac{\vvec_0}{\gammac}&\left(1-\e^{-\gammac t}\right)+
         \left[-\frac{1}{\gammac m}\e^{-\gammac\tc{green}{t}}\int_{0}^{\rd{\tc{green}{t}}}\e^{\gammac t''}\Fvecst\diff t''-(-\tc{green}{0})\right]\nalign
    &+\frac{1}{\gammac m}\rd{\int\limits_0^{t}}\Fvecst\diff\rd{t'}\nalign
      =&\ul[ulc3]{\frac{\vvec_0}{\gammac}\left(1-\e^{-\gammac t}\right)+\frac{1}{\gammac m}\int_{0}^{t}\left(1-\e^{-\gammac(t-t')}\right)\Fvecst\diff t'}
  \end{align*}
\end{sectionbox}
\begin{sectionbox}[\ctr{\scalebox{0.7}{$\obs{\abs{\rvec(t)-\rvec_0}^2}$}}]\nospacing
  \begin{align*}
    &\obs{\abs{\ul[ulc3]{\rvec(t)-\rvec_0}}^2}\eqs{\cref{eq:QuadratVonIntegralen}}\frac{\vvec_0^2}{\gammac^2}\left(1-\e^{-\gammac t}\right)^2+\frac{1}{\gammac^2 m^2}\int\limits_{0}^{t}\int\limits_{0}^{t}
    \left[\phantom{\obs{\left(1-\e^{-\gammac(t-t')}\right)\Fvecst\cdot\left(1-\e^{-\gammac(t-t')}\right)\Fvecst}}\right.\nalign
    &\left.\obs{\left(1-\e^{-\gammac(t-t')}\right)\Fvecst_{'}\cdot\left(1-\e^{-\gammac(t-t'')}\right)\Fvecst_{''}}\right]\diff t'\diff t''\nalign
    &\eqs{\cref{eq:Aquipartitionstheorem}}\ldots+\frac{1}{\gammac^2 m^2}\int\limits_{0}^{t}\int\limits_{0}^{t}
      \big[\ldots\underbrace{\obs{\Fvecst(t')\cdot\Fvecst(t'')}}_{\mathclap{\eqs{\cref{eq:autokorrelationsfunktion}}6\gammac m\kb\Tz\delta(t'-t'')}}\big]\diff t'\diff t''\nalign
    &\eqs{t'=t''}\frac{\vvec_0^2}{\gammac^2}\left(1-\e^{-\gammac t}\right)^2
      +\frac{6\gammac m\kb\Tz}{\gammac^2 m^2}\int\limits_{0}^{t}\left(1-\e^{-\gammac(t-t')}\right)^2\diff t'\nalign
    &=\ldots+\frac{6\kb\Tz}{\gammac m}\int\limits_{0}^{t}\left[1-2\e^{-\gammac(t-t')}+\e^{-2\gammac(t-t')}\right]\diff t'\nalign
    &=\ldots+\frac{6\kb\Tz}{\gammac m}\left.\left[t'+\frac{2}{\gammac}\e^{-\gammac(t-t')}-\frac{1}{2\gammac}\e^{-2\gammac(t-t')}\right]\right\rvert_{0}^t\nalign
    &=\ldots+\frac{6\kb\Tz}{\gammac m}\left[t+\frac{2}{\gammac}-\frac{1}{2\gammac}-\frac{2}{\gammac}-\e^{-\gammac t}+\frac{1}{2\gammac}\e^{-2\gammac t}\right]\nalign
    &=\frac{\vvec_0^2}{\gammac^2}\left(1-\e^{-\gammac t}\right)^2+\frac{3\kb\Tz}{\gammac^2 m}\left[2\gammac t+3-4\e^{-\gammac t}+\e^{-2\gammac t}\right]
  \end{align*}
  Damit folgt dann:
  \begin{empheq}[box=\fbox]{align}\nospacing
    &\overline{\obs{\left(\rvec(t)-\rvec_0\right)^2}}=\lim_{t\to\infty}\obs{\left(\rvec(t)-\rvec_0\right)^2}=\frac{6\kb\Tz}{m\gammac}t\bdlla{=}{\normalfont{phänomenologische Thr.}}6\mcc[D]t\nalign
    &\mcc[D]\equiv\frac{\kb\Tz}{m\gammac}\qquad \rdb{\text{Diffusionskoeffizient}}\nonumber
  \end{empheq}
  \vspace{-1cm}
\end{sectionbox}
\begin{sectionbox}[Phänomenologische Diffusionstheorie]\nospacing
  \begin{numberlist}
      \item \rdb{1. Ficksches Gesetz}:\\ Flussdichte durch Konzentrationsgradienten.
        \begin{align}
          \bdla{\vec{J}}{\textnormal{FLussdichte}}(\rvec,t)=-\mcc[D]\underbrace{\nabla\tura{\mcc}{\textnormal{Konzentration}}(\rvec,t)}_{\mathclap{\text{Konzentrationsgadietn}}}
        \end{align}
      \item \rdb{Kontinuitätsgleichung}:\\ Differentielle Massen/Stofferhaltung.
        \begin{align}
          \difrac{\mcc(\rvec,t)}{t}=-\divg\vec{J}=-\nabla\cdot\vec{J}
        \end{align}
      \item \rdb{2. Ficksches Gesetz/Diffusionsgleichung}:\\ Flussdichte durch Konzentrationsgradienten.
        \begin{align}
          \pfrac{\mcc(\rvec,t)}{t}=-\nabla\cdot \left(-\mcc[D]\nabla\mcc\right)=\mcc[D]\Delta\mcc
        \end{align}
  \end{numberlist}
\end{sectionbox}
\begin{notebox}[Nebenbemerkung]
  Damit kann der atomare Reibungskoeffizient $\gammac$ \cref{eq:atomarerReibungskoeffizient} also entweder experimentel bestimmt werden oder berechnet werden.
\end{notebox}
\subsection{Algorithmus der Langerindynamik $\Wpot\ne0$}
\begin{sectionbox}\nospacing
  Bisher wurde $\Fvec=-\nabla\Wpot=0$ also Null betrachtet um eine analytische Lösung zu finden.\\
  Nun betrachten wir den Fall $\Fvec=-\nabla\Wpot\neq0$. Eine Lösung kann hier nur noch nummerisch gefunden werden.
  \begin{align*}
    m\dot{\vvec}+\underbrace{m\gammac\vvec}_{\mathclap{\text{ratio}}}=\rd{\Fvec}+\underbrace{\Fvecst}_{\mathclap{\text{actio}}} 
  \end{align*}
  Anpassen des Ansatzes \cref{eq:AnsatzWO} mit $t_0<t$ liefrt dann:
  \todo[inline]{Wie genau?}
  \begin{align}
    &\circledItem{1}\nalign
    &\vvec(t)=\vvec_{0}\e^{-\gammac (t-t_0)}+\frac{\e^{-\gammac t}}{m}\int_{t_0}^{t}\e^{\gammac t'}\left(\Fvec(t')+\Fvecst(t')\right)\diff t'\nonumber\nalign
    &\circledItem{2}\quad \rvec(t)=\rvec(t_0)+\int_{t_0}^{t}\vvec(t')\diff t'
  \end{align}
\end{sectionbox}
\begin{sectionbox}[Solution via Leap Frog]\nospacing
  \todo[inline]{Add and understand pluging in}
\end{sectionbox}
\subsection{Fluktuationen}
\begin{defnbox}\nospacing
  \begin{defn}[Varianz]
    Ist der Mittelwert der quadrierten Abweichung einer Observablen $O$ von ihrem Mittelwert $\obs{O}$:
    \begin{align}
      \obs{\left(O-\obs{O}\right)^2}&=\obs{O^2-2O\obs{O}+\obs{O}^2}\nalign
                                      &=\obs{O}^2-2\obs{O}\obs{O}+\obs{O}^2=\boxed{\obs{O^2}-\obs{O}^2}\nonumber
    \end{align}
  \end{defn}
\end{defnbox}
\begin{defnbox}\nospacing
  \begin{defn}[Fluktuationen]
    Sind Zufällige Abweichungen von Systemeingenschaften vom Durchnschnitt, eines Systems im Gleichgewicht.\\
    Als Mass für die Fluktuation einer Observable $O$ kann die Varianz dienen.
    \begin{align}
      \text{\rdb{Fluktuation}}\equiv\obs{O^2}-\obs{O}^2
    \end{align}
  \end{defn}
\end{defnbox}
\begin{sectionbox}[Kanonisches Ensemble]\nospacing
  $(\Nz,\Vz,\Tz)=$konstant$\Rightarrow \Ez$ darf flukturieren.
  \begin{notebox}[Recall]\nospacing
    \begin{align*}
      &\ul[ulc2]{\Zzk}=\sum_{\idxr}e^{-\betac\Ezr}&&\text{\imp{und}}&&\pbr=\frac{1}{\Zzk}\cdot\e^{-\betac\Ezr}\nalign
    \end{align*}
    \begin{flalign}
      &\text{\imp{Mit}}&\ul{\Upot(\Zzk)}=\sum_{\idxr}\pbr\Ezr=-\pfrac{\ln\Zzk}{\betac}\label{eq:kuH3}
    \end{flalign}
  \end{notebox}
  \imp{Studiere}: die Änderung der inneren Energie mit der Temperatur $\Tz\propto\betac$:
  \begin{align*}
    \uldotted{\pfrac{(-\ul{\Upot})}{\betac}}=&\frac{\partial^2\ln\Zz}{\partial\betac^2}=\frac{\partial}{\partial\betac}\left(\pfrac{\ln\Zz}{\betac}\right)\nalign
    \eqs{\text{\rd{C.R.}}}&\frac{\partial}{\partial\betac}\left(\frac{1}{\Zz}\pfrac{\Zz}{\betac}\right)\eqs{\text{\rd{P.R.}}}\left(\frac{\partial}{\partial\betac}\frac{1}{\Zz}\right)\pfrac{\Zz}{\betac}+\frac{1}{\Zz}\frac{\partial^2\Zz}{\partial\betac^2}\nalign
    \eqs{\substack{\cref{eq:kuH1}\\\cref{eq:kuH2}}}&-\frac{1}{\Zz^2}\left(\pfrac{\Zz}{\betac}\right)^2+\frac{1}{\Zz}\sum_{\idxr}\Ezr^2\e^{-\betac\Ezr}\nalign
    \eqs{\cref{eq:kuH3}}&-\frac{1}{\Zz^2}\left(\pfrac{\Zz}{\betac}\right)^2+\sum_{\idxr}\pbr\Ezr^{\rd{2}}\nalign
    \eqs{\substack{\text{Revers
    \rd{C.R.}}\\\cref{eq:ensembelDurchschnitt}}}&-\left(\frac{\partial\ln\Zz}{\partial\betac}\right)^2+\obs{\Ez^{\rd{2}}}\\
    \eqs{\substack{\cref{eq:kuH3}\\\cref{eq:ensembelDurchschnitt}}}&-\left(-\obs{\Ez}\right)^2+\obs{\Ez^2}=\obs{\Ez^2}-\obs{\Ez}^2
  \end{align*}
  Wir können allerdings auch einen anderen Ausdruck ableiten:
  \begin{align*}
    \pfrac{}{\betac}\left(-\obs{\Ez}\right)=&\uldotted{\pfrac{(-\ul{\Upot})}{\betac}}=-\pfrac{\Upot}{\frac{1}{\kb\Tz}}=-\kb\pfrac{\Upot}{\Tz}\pfrac{\Tz}{\Tz^{-1}}\nalign
    =&-\kb\pfrac{\Upot}{\Tz}\pfrac{\Tz}{\Tz^{-1}}=-\kb\pfrac{\Upot}{\Tz}\pfrac{(\Tz^{-1})^{-1}}{\Tz^{-1}}\nalign
       \eqs{\text{\rd{C.R.}}}&\frac{\kb}{\Tz^{-2}}\pfrac{\Upot}{\Tz}\pfrac{\Tz^{-1}}{\Tz^{-1}}=\frac{\kb}{\Tz^{-2}}\pfrac{\Upot}{\Tz}=\kb\Tz^2\pfrac{\Upot}{\Tz}\nalign
    \eqs{\substack{\text{phän. Thermod.}\\\cref{eq:WarmekapazitatVkonst}}}&\kb\Tz^2\cv
  \end{align*}
  Gleichsetzen der beiden Ausdrücke liefert dann:
  \begin{empheq}[box=\widefbox]{equation}
    \obs{\Ez^2}-\obs{\Ez}^2=\kb\Tz^2\cv
  \end{empheq}
\end{sectionbox}  
\begin{notebox}[Bemerkungen]
  \begin{align}
    \frac{\partial^2\ul[ulc2]{\Zz}}{\partial\betac^2}\eqs{\text{\rd{C.R.}}}\pfrac{}{\betac}\left(-\sum_{\idxr}\Ezr\e^{-\betac\Ezr}\right)\eqs{\text{\rd{C.R.}}}\sum_{\idxr}\Ezr^2\e^{-\betac\Ezr}\label{eq:kuH1}
  \end{align}
  \begin{align}
    \frac{\partial\ul[ulc2]{\Zz}^{-1}}{\partial\betac}=\pfrac{}{\betac}\left(\sum_{\idxr}\Ezr\e^{-\betac\Ezr}\right)^{-1}\eqs{\text{\rd{C.R.}}}=-\Zz^{-2}\pfrac{\Zz}{\betac}\label{eq:kuH2}
  \end{align}
\end{notebox}
\begin{notebox}[Nebenbemerkung]
  \begin{numberlist}
      \item Die Fluktuation der inneren Energie nimmt also mit steigender Temperatur zu, was enläuchtend ist.
      \item $\obs{\Ez^2}-\obs{\Ez}^2$ kann mittels Computersimulationen bestimmt werden z.B. bei gefährlichen Stoffen.
      \item $\cv$ kann aber auch Experimentel bestimmt werden.
  \end{numberlist}
\end{notebox}
\begin{sectionbox}[Isotherm-isobares Ensemble]\nospacing
  $(\Nz,\pz,\Tz)=$konstant$\Rightarrow \Vz$ darf flukturieren.
  \begin{empheq}[box=\widefbox]{equation}
    \obs{\Vz^2}-\obs{\Vz}^2=\kb\Tz^2V\bdla{\kappa}{\normalfont{Isotherme Kompressibilität}}
  \end{empheq}
\end{sectionbox}  
\begin{sectionbox}[Grosskanonisches Ensemble]\nospacing
  $(\muz,\Vz,\Tz)=$konstant$\Rightarrow \Nz$ darf flukturieren.
  \begin{empheq}[box=\widefbox]{equation}
    \obs{\Nz^2}-\obs{\Nz}^2=\kb\Tz^2N\bdla{\kappa}{\normalfont{Isotherme Kompressibilität}}\tula{\rho}{\normalfont{Teilchendichte}}
  \end{empheq}
\end{sectionbox}  
\todo[inline]{Add Herleitungen}

%%% Local Variables:
%%% mode: latex
%%% TeX-master: "../formularySPCS"
%%% End:
