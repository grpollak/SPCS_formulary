\subsection{Zustandsumme/Partizipationsfunktion}
\begin{defnbox}\nospacing
  \begin{defn}[Zustandssumme]\leavevmode 
    \begin{numberlist}
        \item Ist ein Mass für die Zahl von Zuständen, die für ein System bei gegebener
    Temperatur thermisch erreichbar sind.
      \item Ist die Summe der Wahrscheinlichkeitsterme über alle Energieniveaus.
      \item Ist notwendig um die Summe der Wahrscheinlichkeiten zu normieren.
      \begin{align*}
        \sum_{\idxr}^{n}\pb_{\idxr}=\sum_{\idxr}^{n}\frac{\e^{-\betac\Ez_{\idxr}}}{\Zz(\Nz,\Vz,\Tz)}=\frac{\Zz(\Nz,\Vz,\Tz)}{\Zz(\Nz,\Vz,\Tz)}=1
      \end{align*}
    \end{numberlist}
  \end{defn}
\end{defnbox}
\begin{notebox}[Eigenschaften]\nospacing
  \begin{numberlist}
        \item Für $\Ez_0=0$ folgt für $\lim\limits_{\Tz\to0}\Zz=1$ $\Rightarrow$ alle Teilchen sind im Grundzustand und der Anteil der Teilchen im
      Angeregten Zustand ist Null.
        \item Für $\Ez_0=0$ folgt für $\lim\limits_{\Tz\to\infty}\Zz=\ul{\infty}$ $\Rightarrow$ jeder Wahrscheinlichkeitsterm der Zustandssummer ist $\e^{-\infty}=1$
      $\Rightarrow$ Zustandssumme ist gleich der Zahl der Molekülzustände (in Makroskopischen Systemen also $\ul{\infty}$) $\iff$ alle Zustände sind erreichbar.
      Angeregten Zustand ist Null.
  \end{numberlist}
\end{notebox}

%%% Local Variables:
%%% mode: latex
%%% TeX-master: "../formularySPCS"
%%% End:
