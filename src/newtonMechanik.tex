\begin{sectionbox}[Darstellung von Bahnkurven]\nospacing
 \begin{equation*}
   \vec{r}=x\vec{e}_x+y\vec{e}_y+z\vec{e}_z:=\vect{x \\ y \\ z}=R 
 \end{equation*} 
\end{sectionbox}
\begin{notebox}[Achtung $\vec{r}\ne R$]
  Dies kann man gut sehen anhand einem gedrehten KS' sehen:
 \begin{equation*}
   \vec{r}=x'\vec{e}'_x+y'\vec{e}'_y+z'\vec{e}'_z:=\vect{x' \\ y' \\ z'}=R' 
 \end{equation*} 
\end{notebox}

%%% Local Variables:
%%% mode: latex
%%% TeX-master: "../formularySPCS"
%%% End:
