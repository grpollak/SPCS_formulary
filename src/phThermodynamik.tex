\begin{defnbox}\nospacing
  \begin{defn}[Zustandsvariablen]
    Grössen die den Zustand eines Systems eindeutig festlegen.
    \[f\stackrel{\text{z.B.}}{=}(E,V,N),(T,V,N),(T,P,N),(S,V,N),\ldots\]
    $\Rightarrow$ Zustandsgrössen müssen wegunabhängig sein $\oint\diff f=0$.
  \end{defn}
\end{defnbox}
\begin{defnbox}
  \begin{defn}[Zustands-funktionen/grössen]
    Sind grössen die alleine durch \rd{Zustandsvariablen} eindeutig bestimmt werden und damit nur vom momentan Zustand abhängen.\\
    \imp{Zustandsgrössen}: p,T,V,n,S,$\underbrace{\text{U},\text{H},\text{A},\text{G}}_{\mathclap{\text{\rd{Thermod. Potentiale}}}},m$
  \end{defn}
\end{defnbox}
\begin{defnbox}
  \begin{defn}[Prozessgrössen]
    Sind wegabhängige Grössen und sind damit keine Zustandsgrössen.\\
    \imp{Prozessgrössen}: $\Delta W$ ,$\Delta Q$,\ldots
  \end{defn}
\end{defnbox}
\begin{sectionbox}[Thermodynamische Schreibweise]\nospacing
  \begin{numberlist}
      \item
    Wert$=\Spot\stackrel{\text{e.g.}}{=}\Spot(\Tz,\Vz)\stackrel{\text{e.g.}}{=}\Spot(\Ez,\Vz)$\\
    \imp{Wichtig}: $\Spot(\Ez,\Vz)$ und $\Spot(\Tz,\Vz)$ sind verschiedene
    Funktionen d.h.\\
    $\Spot=\Spot(\Ez,\Vz)=f(\Ez,\Vz)=\Spot(\Tz,\Vz)=g(\Tz,\Vz)$.
      \item Partielle Ableitung
        \begin{align*}
          \frac{\partial \Spot(\Ez,\Vz)}{\partial \Ez}:=\left(\frac{\partial \Spot}{\partial \Ez}\right)_{\Vz}
        \end{align*}
      \item Totales Differential von Zustandsfunktionen $f$
        \begin{align*}
          &\diff f=\pfrac{f(x,y)}{x}\diff x+\pfrac{f(x,y)}{y}\diff y\nalign
          &=\left(\pfrac{f}{x}\right)_y\diff x+\left(\pfrac{f}{y}\right)_x\diff y=:\mca(x,y)\diff x+\mcb(x,y)\diff y
        \end{align*}
        Aus der zweimaligen Differenzierbarkeit von $f(x,y)$ und dem \rd{Satz von Schwarz} folgt:
        \begin{notebox}
          \begin{align*}
            \frac{\partial^2f(x,y)}{\partial x\partial y}=\frac{\partial^2f(x,y)}{\partial y\partial x}&&\Leftrightarrow&&
          \left(\pfrac{\mca}{y}\right)_x=\left(\pfrac{\mcb}{x}\right)_y\nalign
          \underbrace{\mca(x,y)\diff x+\mcb(x,y)\diff y}_{\text{vollstd. Differnzial}}&&\Leftrightarrow&&
           \left(\pfrac{\mca}{y}\right)_x=\left(\pfrac{\mcb}{x}\right)_y             
          \end{align*}
        \end{notebox}
  \end{numberlist}
\end{sectionbox}
\begin{defnbox}
  \begin{defn}[Phase]
    Ist ein Stoff (=\rd{reine Phase}) oder Stoffgemisch
    (=\rd{Mischphase}) in dem es keine Trennflächen zwischen
    makroskopischen Teilen des System gibt, an denen sich
    Eigenschaften und Zusammensetzung voneinander
    unterscheiden$\Rightarrow$räumlich konst. Eigenschaften.
  \end{defn}
\end{defnbox}
\begin{defnbox}\nospacing
  \begin{defn}[Homogenes System]
    System das nur aus einer Phase besteht.
  \end{defn}
\end{defnbox}
\begin{defnbox}
  \begin{defn}[Heterogenes System]
    System das aus mehrern Phasen besteht.
  \end{defn}
\end{defnbox}
\begin{defnbox}
  \begin{defn}[Offenes System]
    Stoffaustausch über die Grenzen des Systems ist möglich.
  \end{defn}
\end{defnbox}
\begin{defnbox}
  \begin{defn}[Geschlossenes System]
    Stoffaustausch über die Grenzen des Systems ist nicht möglich.
  \end{defn}
\end{defnbox}
\begin{defnbox}
  \begin{defn}[Abgeschlossenes/Isoliertes System]\label{defn:abgSys}
    System das weder Materie (=geschlossenes System) noch Energie mit seiner Umgebung austauschen kann.
  \end{defn}
\end{defnbox}
\begin{defnbox}\nospacing
  \begin{defn}[Extensive Grössen]\nospacing
    Sind propartional zur Grösse des Systems, also zu den Stoffmengen des Systems.\\
    \imp{Geg.}: $\mca+\mcb=$ homogenes System und $f$ eine Zustandsgrösse.
    \begin{align*}
      &f=f_{\mca}+f_{\mcb}&\Longleftrightarrow&&f&\qquad \text{\rd{extensiv}}
    \end{align*}
    \imp{E.g.}: $m, \Vz, \Gpot, n, \Upot, \Spot,\ldots$
  \end{defn}
\end{defnbox}
\begin{defnbox}
\begin{defn}[Intensive Grössen]\nospacing
  Bleiben bei Änderung der Systemrösse, unter sonst gleichen Bedingungen konstant.
  \begin{align*}
    &f=f_{\mca}=f_{\mcb}&\Longleftrightarrow&&f&\qquad \text{\rd{intensiv}}
  \end{align*}
  \imp{E.g.}: T, P, Energiedichte,\ldots
\end{defn}
\end{defnbox}
\begin{notebox}[Bemerkung]\nospacing
\begin{numberlist}
    \item Mit Ausnahme des Volumens werden extensive Grössen i.d.R. klein geschrieben.
    \item Extensive Grössen können in intensive Grössen umgewandelt werden.
  \begin{align}
    f_{\text{\rd{int.}}}=\frac{f_{\text{\rd{ext.}}}}{N_{\text{Sys.}}}\label{eq:extensiveTointensive}
  \end{align}
\end{numberlist}
\end{notebox}
\subsection{Thermodynamische Zustandsgleichungen}
  \label{subsec:Thermodynamische_Zustandsgleichungen}
\begin{sectionbox}\nospacing
  Setzen die Zustandsgrössen $\pz,\Vz,\Tz,\nz$ zueinander in Beziehung.
\end{sectionbox}
\begin{lawbox}\nospacing
  \begin{law}[Ideales Gasgesetz]
    \begin{align}
      \pz=\frac{\nz\Rc\T}{\Vz}
    \end{align}
  \end{law}
\end{lawbox}
\begin{notebox}[Annahmen]
  \begin{numberlist}
      \item Gasteilchen werden als starre Kügelchen angesehen ($\corresponds$ nicht verformbar).
      \item \imp{Keine W.W. zwischen Teilchen}: da die Bewegunsenergie der Teilche viel grösser ist als die
    zwischenmolekulare Kräfte (=elektr. W.W.).
      \item \imp{Betrachtung der Teilchen als Punktmassen}: da der Abstand der Teilchen gegenüber ihrem Abstand
    zueinander und dem zu verfügung stehendem Volumen sehr klein ist.
      \item Die Zusammenstöße der Teilchen miteinander und mit der Wand sollen vollkommen elastisch sein, d.h. es geht dabei keinerlei Energie verloren. 
  \end{numberlist}
\end{notebox}
\begin{notebox}[Gültigkeitsbereich]
  \begin{numberlist}
      \item Für hohe Temperaturen $\rd{\Tz\uparrow}$ und kleine Drücke $\rd{\pz\downarrow}$, da hierdurch das Volumen
    gross wird.
      \item Vorallem für Wasserstoff und Edelgase, aufgrund des kleinen Radius und der ``unverformbarkeit''.
  \end{numberlist}
\end{notebox}
\begin{lawbox}\nospacing
  \begin{law}[Van der Waals Gleichung]
    \begin{align}
      &\pz=-\frac{\mca}{\nz^2}+\frac{\Rc\Tz}{\nz-\mcb}&\text{mit}&& \mca,\mcb\quad \text{Stoffabhängige Parameter}
    \end{align}
  \end{law}
\end{lawbox}
\begin{notebox}[Verbesserung]
  Beachtet Eigenvolumen der Teilchen und V.d.W. Wechselwirkungen realer Gase.
\end{notebox}
% Kalorisch. Zustandsgleichungen
% ------------------------------------------------------------------------------ 
\subsection{Kalorische Zustands-/Energiegleichungen}
\label{subsec:Kalorische_Zustandsgleichungen/Energiegleichungen}
\begin{defnbox}
  \begin{defn}[Thermodynamische Potentiale]
    Sind Zustandsgrössen \imp{mit der dimension Energie}, die das Verhalten thermodynamischer Systeme im Gleichgewicht
    vollständig beschreiben.\\
    \imp{Diese sind}: $\Spot,\Upot,\Hpot,\Apot,\Gpot$ und das grosskanonische Ensemble $\Opot$.
  \end{defn}
\end{defnbox}
\begin{sectionbox}\nospacing
  \imp{Problem}: beschreibung des inneren energetischen Zustands/\rd{thermodynamische potential} eines Systems ist
  allein durch die \textcolor{section}{thermodynamischen Zustandsgleichungen} nicht möglich.\\
  $\Rightarrow$ Kalorische Zustandsgleichungen:
  \begin{align}
    &\boxed{\Upot=\Upot(\Tz,\Vz)}&&\boxed{\Hpot=\Hpot(\Tz,\pz)}
  \end{align}
\end{sectionbox}
\begin{sectionbox}[\subsubsection{Innere/Interne Energie $\Upot$}]\nospacing
    Innere Energie eines Systems ist bestimmt durch:
    \begin{numberlist}
      \item  $E_{\text{kin}}$ der Teilchen.
      \item  $E_{\text{pot}}$ der Systembestandteile.
      \item  $E_{\text{vib}}$ und $E_{\text{rot}}$ der Teilchen.
      \item Die Energie der chemischen Bindung der Teilchen.
    \end{numberlist}
\end{sectionbox}
\begin{lawbox}\nospacing
  \begin{law}[1. Hauptsatz der Thermodynamik \romanNumber{1}]
    Die innere Energie $\Upot$ eines \rd{isolierten} Systems ist konstant.
  \end{law}
\end{lawbox}
\begin{lawbox}\nospacing
  \begin{law}[1. Hauptsatz der Thermodynamik \romanNumber{2}]
    Bei einem System $\Upot_{\text{Sys}}$, dass mit seiner Umgebung $\Upot_{\text{Umg}}$ in Kontakt steht, muss die
    Gesamtenergie $\Upot_{\text{Tot}}$ erhalten bleiben.
    \[\Updownarrow\]
    \begin{align}
      &\Delta\Upot_{\text{Tot}}=\Delta\Upot_{\text{Sys}}+\Delta\Upot_{\text{Umg}}=0
      &\Delta\Upot_{\text{Sys}}=-\Delta\Upot_{\text{Umg}}
    \end{align}
  \end{law}
\end{lawbox}
\begin{defnbox}\nospacing
  \begin{defn}[Wärmekapazität \tc{black}{$\Vz=$\normalfont{const}}]
    \begin{align}
      \diff\qp\stackrel{\Vz=\text{const}}{=}\uldotted{\left(\pfrac{\Upot}{\Tz}\right)_{\Vz}\diff\Tz}&&\cv:=\left(\pfrac{\Upot}{\Tz}\right)_{\Vz}\label{eq:WarmekapazitatVkonst}
    \end{align}
    $\cv$ beschreibt also das Verhältnis von zugeführter Wärme vs Temperaturänderung bei konstantem Druck:
    \begin{align}
      \cv=\frac{\diff\qp}{\diff\Tz}&&\text{ oder }&& \diff\qp=\cv\diff\Tz
    \end{align}
  \end{defn}
\end{defnbox}
\begin{notebox}[\rd{Molare Wärmekapazität}]\nospacing
        \begin{align}
          \cvm=\cv/\m
        \end{align}
\end{notebox}
\begin{notebox}[\rdb{Spezifische Wärmekapazität}]\nospacing
        \begin{align}
          \cvm=\cv/\nz
        \end{align}
\end{notebox}
\begin{lawbox}\nospacing
  \begin{law}[1. Hauptsatz der Thermodynamik \romanNumber{3}]
    In einem \rd{geschlossenen} System, in dem keine chemische Reaktionen oder Phasenübergänge stattfinden
    besteht die Änderung der inneren Energie aus einer Änderung der Wärme $\qp$, Arbeit $\Wp$ oder kombination von beidem.
    \[\Updownarrow\]
    \begin{flalign}
      \diff\Upot&=\tura{\diff\qp}{\normalfont{Heat added \rd{to} the system}}+
      \bdra{\diff\Wp}{\normalfont{Work done \rd{on} Syst.}}
      \qquad=\qquad\diff\qp-\tura{\diff\Wp}{\normalfont{Work done \rd{by} Syst.}}&
    \end{flalign}
    \begin{flalign*}
      &\text{Mit}& \Wp&=\Wp_{\text{Vol}}+\Wp_{\text{Elekt}}=-\pz\diff\Vz+\Wp_{\text{Elekt}}&\nalign
      &\text{Tot. Differential}&\diff\Upot&=\uldotted{\left(\pfrac{\Upot}{\Tz}\right)_{\Vz}\diff\Tz}+\left(\pfrac{\Upot}{\Vz}\right)_{\Tz}\diff\Vz\nalign
      &&&=\diff\qp-\pz\diff\V\nalign
      &&&\eqs{\normalfont{\cref{eq:Entropie}}}\Tz\diff\Spot-\pz\diff\Vz\nalign
      &&&=\left(\pfrac{\Upot}{\Spot}\right)_{\Vz}\diff\Spot+\left(\pfrac{\Upot}{\Vz}\right)_{\Tz}\diff\Vz
    \end{flalign*}
  \end{law}
\end{lawbox}
\begin{defnbox}\nospacing
  \begin{defn}[Fundamentalgleichung]
    Für eine \rd{rev} Zustandsänderung ($\diff\qp_{\text{rev.}}=\Tz\diff\diff\Spot$) in einem \rd{geschlossenen} System gitl:
    \begin{align}
      \diff\Upot=\Tz\diff\Spot-\pz\diff\Vz
    \end{align}
  \end{defn}
\end{defnbox}
\begin{sectionbox}[\subsubsection{Wärmekapazität \tc{black}{$\Vz=$\normalfont{const}}}]\nospacing
  Betrachten wir die Änderung der inneren Energie $\Upot$ bei konstantem Volumen $\partial\Upot/\partial\Vz=0$, so fällt
  uns auf dass keine Volulmenarbeit verichtet wird:\\ $-\pz\diff\Vz=0$ da $\diff\Vz=0$ $\Rightarrow W=0$. 
  \begin{empheq}[box=\fbox]{align*}
    \Rightarrow \diff\Upot=\diff\qp&&\imp{\text{für}}&& \Vz=\text{const}\imp{\text{ und }} \Wp_{\text{elek}}=0
  \end{empheq} 
  Vergleichen wir nun mit dem totalen differential von $\Upot$, so folgt wieder \cref{eq:WarmekapazitatVkonst}
\end{sectionbox}
\subsubsection{Verschiedene Arten von Prozessen}
\label{subsec:Verschiedene_Arten_von_Prozessen}
\begin{defnbox}
  \begin{defn}[Isothermer Prozess]\nospacing
    \begin{equation}
      \Tz=\text{const}
    \end{equation}
  \end{defn}
\end{defnbox}
\begin{defnbox}
  \begin{defn}[Isochorer Prozess]\nospacing
    \begin{align}
      &\Vz=\text{const}
    \end{align}
  \end{defn}
\end{defnbox}
\begin{defnbox}
  \begin{defn}[Isobarer Prozess]\nospacing
    \begin{align}
      &\pz=\text{const}
    \end{align}
  \end{defn}
\end{defnbox}
\begin{defnbox}
  \begin{defn}[Adiabatischer Prozess]\nospacing
    \begin{equation}
      \qp=\text{const}
    \end{equation}
  \end{defn}
\end{defnbox}
\begin{notebox}[Verschiedene Arten von Arbeit]\nospacing
  \begin{numberlist}
      \item Expansion gegen konst. Druck=Isobarer Prozess:
        \begin{equation*}
        W=-\pz_{\text{ext}}\int_{\Vz_{\text{E}}}^{\Vz_{\text{A}}}\diff\Vz=-\pz_{\text{ext}}(\Vz_{\text{E}}-\Vz_{\text{A}})
        \end{equation*}
      \item Reversible isotherme Expansion:
        \begin{align*}
          W&=-\int_{\Vz_{\text{E}}}^{\Vz_{\text{A}}}\pz(\Vz,\Tz)\diff\Vz\stackrel{\text{i.d.G.}}{=}-\nz\Rc\Tz\int_{\Vz_{\text{E}}}^{\Vz_{\text{A}}}\frac{\diff \Vz}{\Vz}\nalign
          &=-\nz\Rc\Tz\ln\frac{\Vz_{\text{E}}}{\Vz_{\text{A}}}
        \end{align*}
          \item Isochore Expansion und freie Expansion ins Vakuum $\pz_{\text{ext}}=0$
        \begin{align*}
          \pz_{\text{ext}}=0&\Rightarrow&\Wp=0
        \end{align*}
  \end{numberlist}
\end{notebox}
\begin{sectionbox}[\subsubsection{Enthalpie $\Hpot$}]\nospacing
  Die meisten Reaktionen im Labor laufen in offenen Gefässen ab, sie laufen also nicht bei constantem Volumen sondern
  unter konstanem Druck ab.\\
  Wir suchen daher eine neues Zustandsfunktionen die für konstanten Druck besonders einfach wird.
\end{sectionbox}
\begin{defnbox}\nospacing
  \begin{defn}[Entahlpie]
    \begin{align}
      \Hpot:=\Upot+\pz\cdot\Vz &&\text{und}&&\diff\Hpot&=\diff\qp+\Vz\diff\pz\nalign
      &&&&&=\Tz\diff\Spot+\Vz\diff\pz\nonumber
    \end{align}
    \begin{align*}
      &\text{Totales Diff.}&&\diff\Hpot=\uldotted[ulc2]{\left(\pfrac{\Hpot}{\Tz}\right)_{\pz}\diff\Tz}+\left(\pfrac{\Hpot}{\pz}\right)_{\Tz}\diff\pz\nalign
      &&&\hphantom{\diff\Hpot}=\left(\pfrac{\Hpot}{\Spot}\right)_{\pz}\diff\Spot+\left(\pfrac{\Hpot}{\pz}\right)_{\Tz}\diff\pz\nalign
      &\pz=\text{const}&&\diff\Hpot=\diff\qp\qquad\text{für}\qquad\Wp_{\text{elek.}}=0
    \end{align*}
  \end{defn}
\end{defnbox}
\begin{notebox}[Bemerkungen]
  \begin{numberlist}
      \item $\Hpot$ ist eine Zustandskuntion da sie aus Zustandsfunktionen zusammengestzt ist.
      \item $\diff\Hpot=\diff\Upot+\diff(\pz\Vz)=\diff\qp-\pz\diff\Vz+\pz\diff\Vz+\diff\pz\Vz$
  \end{numberlist}
\end{notebox}
\begin{sectionbox}[\subsubsection{Wärmekapazität\tc{black}{$\pz=$\normalfont{const}}}]\nospacing
  Betrachten wir die Änderung der Enthalpie $\Hpot$ bei konstantem Druck $\partial\Hpot/\partial\pz=0$, so fällt
  uns auf dass : $-\Vz\diff\pz=0$.
  \begin{empheq}[box=\fbox]{align*}
    \Rightarrow \diff\Hpot=\diff\qp&&\imp{\text{für}}&& \pz=\text{const}\imp{\text{ und }} \Wp_{\text{elek}}=0
  \end{empheq} 
  Vergleichen wir nun mit dem totalen differential von $\Hpot$, so folgt:
\end{sectionbox}
\begin{defnbox}\nospacing
  \begin{defn}[Wärmekapazität \tc{black}{$\pz=$\normalfont{const}}]
    \begin{align}
      \diff\qp\stackrel{\pz=\text{const}}{=}\uldotted[ulc2]{\left(\pfrac{\Hpot}{\Tz}\right)_{\pz}\diff\Tz}&&\cp:=\left(\pfrac{\Hpot}{\Tz}\right)_{\pz}
    \end{align}
    $\cp$ beschreibt also das Verhältnis von zugeführter Wärme vs Temperaturänderung bei konstantem Druck:
    \begin{align}
      \cp=\frac{\diff\qp}{\diff\Tz}&&\text{ oder }&& \diff\qp=\cp\diff\Tz
    \end{align}
  \end{defn}
\end{defnbox}
\subsection{Spontanität von Prozessen}
\begin{defnbox}\nospacing
  \begin{defn}[Reversibler Prozess]
    \begin{numberlist}
        \item Sind \rd{reibungslos}, da Reibung Wärme produziert.
        \item \rd{Quasi Statisch} $=$ Prozess befindet sich immer so gut wie im Glg, als im Quasigleichgewicht.
    \end{numberlist}
  \end{defn}
\end{defnbox}
\begin{notebox}[Bemerkung]
  In der Makrowelt gibt es keine wirklichen reversiblen Prozesse.
\end{notebox}
\begin{lawbox}\nospacing
  \begin{law}[2. Hauptsatz der Thermodynamik]\label{defn:2Hauptsatz}
    Es gibt keine Zustandsänderung, deren einziges Ergebnis die Übertragung von Wärme $\qp$ von von einem Körper
    niederer Temperatur auf einen Körper höherer ist. 
    \[\Updownarrow\]
    \begin{align}
      \diff\Spot_{\text{Syst.}}>0\qquad&\text{für \rd{spontane} Prozesse in \rd{abgeschlossenen}}& \nonumber\nalign
      &\text{Systemen}&
    \end{align}
  \end{law}
\end{lawbox}
\begin{proofbox}
  \begin{proof}
    Für abgeschlossene Systeme \cref{defn:abgSys} gilt $\diff\qp=0$.\\
    Mit der Clausiussche Ungleichung \cref{defn:Cugl} folgt dann sofort der 2. HS.
  \end{proof}
    \end{proofbox}
\begin{defnbox}\nospacing
  \begin{defn}[Entropie]
    Da Wärme eine Prozessgrösse ist definieren wir eine neue Zustandsfunktion, die den 2. Hauptsatz der Thermodynamik
    \cref{defn:2Hauptsatz} mathematisch beschreibt:
    \begin{align}
      \diff\Spot_{\text{Syst.}}=\frac{\diff\qp_{\text{rev.}}}{T}\label{eq:Entropie}
    \end{align}
  \end{defn}
\end{defnbox}
\begin{sectionbox}[\subsubsection{Clausiussche Ungleichung}\label{subsubsec:Clausische Ungleichung}]\nospacing
  \ul[ulc4]{Unter \rd{reversiblen} Prozessbedingunen wird mehr Arbeit}\\
  \ul[ulc4]{verrichtet als unter \rd{irreversiblen}.}\\
  Dies ist, da Arbeit eine Prozessgrösse ist und in der Form von Wärme verloren geht. \ul[ulc3]{Da die innere Energie eine
  Zustandsfunktion ist gilt}:
  \begin{align*}
    &&\diff\Upot=\diff\qp+\diff\Wp\tc{ulc3}{=}\diff\qp_{\text{rev.}}+\diff\Wp_{\text{rev.}}\nalign
    \Rightarrow&&\diff\qp_{\text{rev.}}-\diff\qp=\diff\Wp-\diff\Wp_{\text{rev.}}\tc{ulc4}{\geq}0\nalign
                  \Rightarrow&&\diff\qp_{\text{rev.}}\geq\diff\qp\Rightarrow \frac{\diff\qp_{\text{rev.}}}{\Tz}\geq\frac{\diff\qp}{\Tz}
  \end{align*}
\end{sectionbox}
\begin{defnbox}\nospacing
  \begin{defn}[Clausiussche Ungleichung]\label{defn:Cugl}
    \begin{align}
        \diff\Spot\geq\frac{\ul{\diff\qp}}{\Tz}
    \end{align}
  \end{defn}
\end{defnbox}
\begin{sectionbox}[Umgebungsentropie]\nospacing
  Die Umgebung entspricht einem Resevoir konstanten Volumens
  $\Rightarrow\Delta _{\text{Umg.}}\stackrel{\Vz=\text{const}}=\Delta\qp_{\text{Umg.}}$\\
  Da die innere Energie eine Zustandsgrösse ist (=Wegunabhängig), hängt sie nicht davon ab ob ein Prozess reversible
  od. irreversible abläuft:
  \begin{align}
    \Delta\Spot_{\text{Umg.}}=\frac{\diff\qp_{\text{Umg.}}}{\Tz}
  \end{align}
\end{sectionbox}
\begin{notebox}[Bemerkungen]
  \begin{numberlist}
      \item Arbeit erfodert eine geordnete Teilchenbewegung.
      \item Um Wämrme von einem System niederer Temperatur $\Tz_1$ zu einem System höherer Temperatur $\Tz_2$
    zuzuführen, müssen wir Arbeit am System verrichten.\\
    Dies erfordert zwar Ordnung und damit eig. eine verminderung der Entropie des Systems, allerdings bleibt die
    Entropie für das ges. System=Syst.+Umgb. für reversible Prozesse konstant.\\
    Für irreversible Prozesse geht arbeit in Form von Reibungswärme verloren $\Rightarrow\diff\Spot>0$.
  \end{numberlist}
\end{notebox}
\begin{lawbox}\nospacing
  \begin{law}[3. Hauptsatz der Thermodynamik\tc{black}{/}Nernst Theorem]\label{defn:2Hauptsatz}
    Es ist nicht möglich ein System bis zum absoluten Nullpunkt abzukühlen.
  \end{law}
\end{lawbox}
\begin{sectionbox}[\subsubsection{Weitere Thermodynamische Potentiale}\label{subsubsec:ThPotentiale}]\nospacing
  Betrachten wir die Clausiussche Ungleichung bei bei konstantem Volumen
  $\diff\qp_{\Vz}\stackrel{V=\text{const}}{=}\ul{\diff\Upot}$ bzw. konstantem Druck $\diff\qp_{\pz}\stackrel{p=\text{const}}{=}\ul{\diff\Hpot}$
  so folgt mit \cref{defn:Cugl}:
  \begin{align}
    \ul{\diff\Upot}-\Tz\diff\Spot\leq0 &&\text{für}&&\Vz=\text{const}\label{eq:cl1}\nalign
    \ul{\diff\Hpot}-\Tz\diff\Spot\leq0 &&\text{für}&&\pz=\text{const}\label{eq:cl2}
  \end{align}
\end{sectionbox}  
\begin{defnbox}\nospacing
  \begin{defn}[Freie/Helmoltz Energie]
    \begin{align}
      \Fpot:&=\Upot-\Tz\Spot
    \end{align}
    \begin{align*}
      &\text{Tot. Differential}&\diff\Fpot:&=\diff\Upot-\diff(\Tz\Spot)\nalign
      &&&=-\pz\diff\Vz-\Spot\diff\Tz\nalign
      &&&=\left(\pfrac{\Upot}{\Vz}\right)_{\Tz}\diff\Vz+\left(\pfrac{\Upot}{\Tz}\right)_{\Vz}\diff\Tz
    \end{align*}
  \end{defn}
\end{defnbox}
\begin{defnbox}\nospacing
  \begin{defn}[Freie Enthalpie/Gibbs Energie]
    \begin{align}
      \Gpot:&=\Hpot-\Tz\Spot
    \end{align}
    \begin{align*}
      &\text{Tot. Differential}&\Gpot:&=\diff\Hpot-\diff(\Tz\Spot)\nalign
      &&&=\Vz\diff\pz-\Spot\diff\Tz\nalign
      &&&=\left(\pfrac{\Gpot}{\pz}\right)_{\Tz}\diff\pz+\left(\pfrac{\Gpot}{\Tz}\right)_{\pz}\diff\Tz
    \end{align*}
  \end{defn}
\end{defnbox}
\begin{sectionbox}[Spontanität von Prozessen]\nospacing
  Mit \cref{eq:cl1,eq:cl2} folgt für die Spontanität von Prozessen
  \begin{align}
    \diff\Fpot(\Tz,\Vz)\leq0&&\text{für}&&\Vz=\text{const}\nalign
    \diff\Gpot(\Tz,\pz)\leq0&&\text{für}&&\pz=\text{const}
  \end{align}
\end{sectionbox}
\begin{notebox}[\rdb{Exothermer Prozess} $\Delta\Hpot>0$]
  Wärme wird an Umgebung abgegeben.
\end{notebox}
\begin{notebox}[\rdb{Endothermer Prozess} $\Delta\Hpot<0$]
  Wärme wird von Umgebung aufgenommen.
\end{notebox}
\begin{notebox}[\rdb{Exergonischer Prozess} $\Delta\Upot>0$]
  Prozess läuft freiwillig ab.
\end{notebox}
\begin{notebox}[\rdb{Endergoner Prozess} $\Delta\Upot<0$]
  Prozess läuft nicht freiwillig ab.
\end{notebox}
\begin{notebox}[Bemerkung]\nospacing
  \begin{align*}
    \diff\Fpot&=\diff\qp_{\text{rev.}}-\pz\diff\Vz-\Tz\diff\Spot-\diff\Tz\Spot\nalign
                \Rightarrow&\diff\Fpot=-\pz\diff\Vz-\Spot\diff\Tz
  \end{align*}
  \begin{align*}
    \diff\Gpot&=\Tz\diff\Spot+\Vz\diff\pz-\Tz\diff\Spot-\diff\Tz\Spot\nalign
                \Rightarrow&\diff\Gpot=\Vz\diff\pz-\Spot\diff\Tz
  \end{align*}
\end{notebox}
\begin{sectionbox}[Relationen]\nospacing
  Aus den totalen Differentialen der einzlnen Potentiale lassen die folgenden Beziehungen ablesen:
  \begin{align}
      \left(\pfrac{\Upot}{\Spot}\right)_{\Vz}&=\Tz&&&\left(\pfrac{\Upot}{\Vz}\right)_{\Spot}&=-\pz\label{eq:r1}\nalign
      \left(\pfrac{\Fpot}{\Tz}\right)_{\Vz}&=-\Spot&&&\left(\pfrac{\Fpot}{\pz}\right)_{\Spot}&=-\pz\label{eq:r2}\nalign
      \left(\pfrac{\Hpot}{\Spot}\right)_{\pz}&=\Tz&&&\left(\pfrac{\Hpot}{\pz}\right)_{\Spot}&=\Vz\label{eq:r3}\nalign
      \left(\pfrac{\Gpot}{\Tz}\right)_{\pz}&=-\Spot&&&\left(\pfrac{\Gpot}{\pz}\right)_{\Tz}&=\Vz\label{eq:r4}
  \end{align}
\end{sectionbox}
\begin{sectionbox}[Maxwell Relationen]\nospacing
  Mittels des Satzes von Schwartz und \cref{eq:r1,eq:r2,eq:r3,eq:r4} lassen sich die so genannten Maxwell Relationen Herleiten z.B. angewendet auf die innere Energie:
  \begin{align*}
    \left[\frac{\partial}{\partial\Vz}\left(\pfrac{\Upot}{\Spot}\right)_{\Vz}\right]_{\Spot}&=\left[\frac{\partial}{\partial\Spot}\left(\pfrac{\Upot}{\Vz}\right)_{\Spot}\right]_{\Vz}\nalign
    \left[\frac{\partial}{\partial\Vz}\Tz\right]_{\Spot}&=\left[\frac{\partial}{\partial\Spot}(-\pz)\right]_{\Vz}
  \end{align*}
\end{sectionbox}
\begin{lawbox}\nospacing
  \begin{law}[Maxwell-Relationen]
  \begin{align}
    \left(\pfrac{\Tz}{\Vz}\right)_{\Spot}=-&\left(\pfrac{\pz}{\Spot}\right)_{\Vz}\nalign
    \left(\pfrac{\Tz}{\pz}\right)_{\Spot}=&\left(\pfrac{\Vz}{\Spot}\right)_{\pz}\nalign
    \left(\pfrac{\Spot}{\Vz}\right)_{\Tz}=&\left(\pfrac{\pz}{\Tz}\right)_{\Vz}=\frac{\betac}{\kappa}\nalign
    -\left(\pfrac{\Spot}{\pz}\right)_{\Tz}=&\left(\pfrac{\Vz}{\Tz}\right)_{\pz}=\Vz\betac
  \end{align}
  \end{law}
\end{lawbox}
\subsection{Das Chemische potential}
\begin{sectionbox}[Einleiung]\nospacing
 Bis jetzt haben wir die Änderung der thermodynamischen Potential nur für
 Systeme mit konstanten Stoffmengn betrachtet. Da chemische Reaktionen aber Reaktanten (=Reaktionspatner) verbrauchen und Produkte erzeugen erfordet es jedoch,
 dass wir die Definitionen der Potentiale Anpassen.
 \begin{align*}
   &\text{z.B.}&\diff\Gpot=&\left(\pfrac{\Gpot}{\pz}\right)_{\Tz,n_1,n_2,\ldots}\diff\pz+\left(\pfrac{\Gpot}{\Tz}\right)_{\pz,n_1,n_2,\ldots}\diff\Tz\nalign
               &&&+\sum_{\idxi}^N\left(\pfrac{\Gpot}{n_{\idxi}}\right)_{\Tz,\pz,n_{\idxj\neq\idxi}}
 \end{align*}
\end{sectionbox}
\begin{defnbox}\nospacing
  \begin{defn}[Partielle Molare Grössen $Y_{\idxi}$]
    Die partielle molare Grösse $Y_{\idxi}$ einer Mischkomponente $\idxi$ ist definiert als die Änderung von $Y$ bei zugabe von 1mol der Komponente $\idxi$,
    bei konstanten anderen Grössen.
    \begin{align*}
      &\text{\imp{Sei}}:& Y=Y(x_1,x_1,\vec{n})&&Y_{\idxi}=\left(\pfrac{Y}{n_{\idxi}}\right)_{x_1,x_2,n_{\idxj\neq\idxi}}
    \end{align*}
  \end{defn}
\end{defnbox}
\begin{defnbox}\nospacing
  \begin{defn}[Chemisches Potential]$\idxj\neq\idxi$
    \begin{align}
      \mupot=\left(\pfrac{\Upot}{n_{\idxi}}\right)_{\Spot,\Vz,n_{\idxj}}&&
      \mupot=\left(\pfrac{\Hpot}{n_{\idxi}}\right)_{\Spot,\pz,n_{\idxj}}\nalign
      \mupot=\left(\pfrac{\Fpot}{n_{\idxi}}\right)_{\Vz,\Tz,n_{\idxj}}&&
      \mupot=\left(\pfrac{\Gpot}{n_{\idxi}}\right)_{\pz,\Tz,n_{\idxj}}&&
    \end{align}
  \end{defn}
\end{defnbox}
\begin{defnbox}\nospacing
  \begin{defn}[Thermodynamischen Potentiale]
    \begin{align}
      \Upot&=\Tz\diff\Spot-\pz\diff\Vz+\mu\diff\Nz\nalign
      \Fpot&=-\Spot\diff\Tz-\pz\diff\Vz+\mu\diff\Nz\nalign
      \Hpot&=\Tz\diff\Spot+\Vz\diff\pz+\mu\diff\Nz\nalign
      \Gpot&=-\Spot\diff\Tz+\Vz\diff\pz+\mu\diff\Nz\label{eq:chemPotG}
    \end{align}
  \end{defn}
\end{defnbox}
\begin{sectionbox}[Duhem-Gibbs-Relation]\nospacing
  Das chemische Potential hat eine besonders einfache Beziehung zur freien Entahlpie $\Gpot=\Gpot(\Tz,\pz,\Nz)$:
  Als extensive Grösse lässt sich $\Gpot$ schreiben als:
  \begin{align*}
    \Gpot(\Tz,\pz,\Nz)\eqs{\cref{eq:extensiveTointensive}}\nz\Gpot_m(\Tz,\pz,\Nz)
  \end{align*}
  Mit $\Gpot=\nz\mupot$ folgt dann für das totale Differential:
  \begin{align*}
    \diff\Gpot=\sum_{\idxi}\nz_{\idxi}\diff\mupot_{\idxi}+\sum_{\idxi}\mupot_{\idxi}\diff\nz_{\idxi}\eqs{\mathclap{\cref{eq:chemPotG}}}-\Spot\diff\Tz+\Vz\diff\pz+\sum_{\idxi}\mupot_{\idxi}\diff\nz_{\idxi}
  \end{align*}
\end{sectionbox}
\begin{defnbox}\nospacing
  \begin{defn}[Duhem-Gibss-Relation]
    \begin{align}
      \sum_{\idxi}\nz_{\idxi}\diff\mupot_{\idxi}=-\Spot\diff\Tz+\Vz\diff\pz\label{eq:DuhemGibbs}
    \end{align}
  \end{defn}
\end{defnbox}
\begin{sectionbox}[Grosskanonisches Potential]\nospacing
  \begin{align*}
              &&\Omega&=\Fpot-\nz\mu=\Upot-\Tz\Spot-\nz\mupot\nalign
              &&\diff\Omega&=-\Spot\diff\Tz-\pz\diff\Vz-\nz\diff\mupot\nalign
    \Rightarrow&&\Omega&=\Omega(\Tz,\Vz,\mupot)
  \end{align*}
\end{sectionbox}
\begin{notebox}[Bemerkungen]\nospacing
  \begin{numberlist}
      \item Aus \cref{eq:DuhemGibbs} folgt das $\Tz$ und $\pz$ die natürlichen Variablen des chemischen Potentials sind:
    \begin{align*}
      \mupot=\mupot(\Tz,\pz)
    \end{align*}
      \item Im Glg. ist das chem. Pot. jedes einzelnen Stoffes der Ganzen Mischung Gleich.
  \end{numberlist}
\end{notebox}
%%% Local Variables:
%%% mode: latex
%%% TeX-master: "../formularySPCS"
%%% End:
