\begin{sectionbox}[Überblick]\nospacing
 \begin{circlelist}
    \item Strukturelle Eigenschaften:
    \begin{numberlist}
        \item Teilchenposition
        \item Radiale Verteilungsposition
        \item Orientierungskorrelationsfunktion
        \item Lösungsmittelzugängliche Flächen\\ (z.B. wohin fliesst das Wasser)
        \item Gyrationradius=Trägheitsradius\\ (radius of gyration)
    \end{numberlist}
    \item Thermodynamische Eigenschaften:
    \begin{numberlist}
        \item Energie/Freie Energie
        \item Wärmekapazität
        \item Entropie
        \item Kompressibilität
        \item Thermodynamischer Ausdehenungskoeffizient
        \item \ldots
    \end{numberlist}
    \item Dynamische Eigenschaften:
    \begin{numberlist}
        \item Diffusion
        \item Viskosität
    \end{numberlist}
    \item Elektrodmagnnetisch Eigenschaften:
    \begin{numberlist}
        \item Dielektrische Permitivität
        \item Spektrokopische Eigenschaften (NMR, IR,\ldots)
    \end{numberlist}
 \end{circlelist}
\end{sectionbox}
\begin{sectionbox}[\subsubsection{Mittlere Atomposition}]\nospacing
  \begin{align}
    \obs{\rveci}=&\frac{1}{t}\int\limits_{0}^t\rveci(t')\diff t'\bda{\approx}{\normalfont{Diskret.}}\frac{1}{N_{t}}\sum_{\idxn=0}^{N_{t}}\rveci(t_{\idxn})\nalign\nonumber
    N_{t}:&\text{Anzahl der Zeitschritte}
  \end{align}
\end{sectionbox}
\begin{notebox}[Nützlichkeit]\nospacing
  Zum Beispiel zur Berechnung von Fluktuationen in Atompositionen:
  \begin{align*}
    \sqrt{\Big\langle\left(\rveci-\obs{\rveci}\right)^2\Big\rangle}=\sqrt{\frac{1}{N_t}\sum_{\idxn=0}^{N_t}\left[\rveci-\obs{\rveci}\right]^2}
  \end{align*}
  $\Rightarrow$ Experimentelle Messgrösse um die Fluktuation zu berechnen.\\
  (Kristallographie: $\betac$-Faktor von Atom $\idxi$)
  \todo[inline]{Was ist hier gemeint?}
\end{notebox}
\begin{sectionbox}[\subsubsection{Radius of Gyration}]\nospacing
  \begin{align*}
    \vec{R}_{\text{Gyr}}=\sqrt{\frac{1}{N_{\mca}}\sum_{\idxn=0}^{N_{\mca}}\left[\rveci(t_{\idxn})-\vec{R}_{\text{CM}}(t_{\idxn})\right]^2}
  \end{align*}
  \begin{align*}
  &\vec{R}_{\text{CM}}(t_{\idxn})=\frac{1}{M}\sum_{\idxn=0}^{N_{\mca}}m_{\idxn}\rveci(t_{\idxn}) &&\text{(Massenschwerpunkt)}\nalign
                                                                      &M=\sum_{\idxn=0}^{N_{\mca}}=m_{\idxn}&&\text{(Massenschwerpunkt)}\nalign
    &N_{\mca}:&&\text{\# Atome im Molekül}
  \end{align*}
\end{sectionbox}
\begin{sectionbox}[\subsubsection{Root Mean Square Atom Positon (RMSD)}]\nospacing
  Ist ein Mass für die Ähnlichkeit zweier Konfigurationen $m$ und $n$:
  \begin{align*}
    \text{RMSD}(m,n)=\sqrt{\frac{1}{N_{\mca}}\sum_{\idxi=1}^{N_{\mca}}\left[\rveci(m)-\rveci(n)\right]^2}
  \end{align*}
\end{sectionbox}
\begin{sectionbox}[\subsubsection{Diffusionskoeffizient}]\nospacing
  \begin{align*}
    \mcc[D]=\lim_{t\to\infty}\frac{\obs{\left[\rvec(t)-\rvec(t_0)\right]^2}}{6t}
  \end{align*}
\end{sectionbox}
\begin{sectionbox}[\subsubsection{Radiale Verteilungsfunktion}]\nospacing
  \imp{Gegeben}: System von $\idxN$ Teilchen in einem Volumen mit Partikelkoordinaten $\rveci,\quad\idxi=1,\ldots,\idxN$. Die potentielle Energie als Resultat
  von Teilchen-Teilchen W.W. ist $\Wpot_{\idxN}(\rvec_1,\ldots,\rvec_{\idxN})$.\\
  Die Wahrscheinlichkeit einer elementaren Konfiguration, also das Auffinden von Partikel 1 in $\diff\rvec_1$, Partikel 2 in $\diff\rvec_2$ usw. ist gegeben durch:
  \begin{align}
    \Pb^{(\idxN)}(\rvec_1,\ldots,\rvec_{\idxN})\diff\rvec_1\cdots\diff\rvec_{\idxN}=\frac{\e^{-\betac\Wpot_{\idxN}}}{\Qzk}\diff\rvec_1\cdots\diff\rvec_{\idxN}
  \end{align}
  Die totale Nummer der Teilchen ist riesig, so dass $\Pb^{(\idxN)}$ nicht sehr nützllich ist. Es ist aber auch möglich die Wahrscheinlichkeit einer
  \rd{reduzierten Konfiguration} zu berechnen, bei der die Position von legdiglich $\idxn<\idxN$ Teilchen $\rvec_1,\ldots,\rvec_{\idxn}$ fixiert ist. Die
  restlichen $\idxN-\idxn$ Teilchen können sich dann frei bewegen. $\Rightarrow$ damit müssen wir dann noch über die restlichen Koordinaten
  $\rvec_{\idxn+1},\ldots,\rvec_{\idxN}$ integrieren:
  \begin{align}
    &\Rightarrow& \ul{\Pb^{(\idxn)}(\rvec_1,\ldots,\rvec_{\idxn})}=\nalign
                  &&=\frac{1}{\Qzk}\int\ldots\int\e^{-\betac\Wpot}\diff_{\idxn+1}\cdots\diff\rvec_{\idxN}\label{eq:Prob}
  \end{align}
  Falls es identische Partikel gibt, ist es von grösserer Interesse die Wahrscheinlichkeit zu berechnen das $\idxn$ gleiche Teilchen die Positionen
  $\rvec_1,\ldots,\rvec_{\idxn}$, (in beliebigen Permutationen) besetzen.
  \begin{align}
    \Rightarrow& \rho^{(\idxn)}(\rvec_1,\ldots,\rvec_{\idxn})=\frac{N!}{(N-n)!}\ul{\Pb^{(\idxn)}(\rvec_1,\ldots,\rvec_{\idxn})}\label{eq:nProb}
  \end{align}
\end{sectionbox}
\begin{defnbox}\nospacing
  \begin{defn}[Korrelations Funktion $\rd{g}^{(\idxn)}$]
    \begin{align}
      \rho^{(\idxn)}(\rvec_1,\ldots,\rvec_{\idxn})=&\rho^{\idxn}\rd{g}^{(\idxn)}(\rvec_1,\ldots,\rvec_{\idxn})
    \end{align}
    \begin{align}
      \rd{g}^{(\idxn)}(\rvec_1,\ldots,\rvec_{\idxn})&=\frac{\idxN!}{(\idxN-\idxn)!}\label{eq:Korrelationsfunktion}\nalign
                                                      &\cdot\frac{1}{\Qzk}\int\ldots\int\e^{-\betac\Wpot}\diff_{\idxn+1}\cdots\diff\rvec_{\idxN}\nonumber
    \end{align}
  \end{defn}
\end{defnbox}
\begin{notebox}[Formale Eigenschaften]
  \begin{numberlist}
      \item Kurz vs. Langreichweitig
        \begin{align}
          &\rd{\text{Kurzreichweitig}}&\lim_{\rvec\to0}\rd{g}(\rvec)=0\nalign
          &\rd{\text{Langreichweitig}}&\lim_{\rvec\to\infty}\rd{g}(\rvec)=1
        \end{align}
      \item 
        \begin{align}
          \iiint_{0}^{\infty}\rd{g}(\rvec)\rho\diff x\diff y\diff
          z=\int_{0}^{\infty}\rd{g}(\rvec)\rho\underbrace{\overbrace{4\pi}^{\mathclap{\int\limits_{0}^{\pi}\int\limits_{0}^{2\pi}\sin\theta\diff\theta\diff\phi}}\rvec^2}_{\mathclap{\diff x\diff y\diff z=r^2\diff r\sin\theta\diff\theta\diff\phi}}\diff\rvec=\idxN-1
        \end{align}
  \end{numberlist}
\end{notebox}
\begin{notebox}[Bemerkungen]
  \begin{numberlist}
      \item $\rho=\frac{V}{N}$ durchschnittliche Dichte.
      \item Falls das System aus sphärischen Koordinaten besteht dann hängt Korrelationfunktion zwischen zwei Partikeln $\rho^{(2)}(\rvec_1,\rvec_{\idxn})$ nur
    von deren Distanz $\rvec_{12}=\abs{\rvec_2-\rvec_1}$ ab.\\
    Sei nun Partikel 1 das Zentrum des Koordinatensystems, so entspricht $\rho\cdot\rd{g}(\rvec)\diff\rvec$ der Durchschnittlichen Zahl der Teilchen (unter den
    verbleibenden $(\idxN-1)$ die im Volumen $\diff\rvec$ um das Zentum $\rvec$ gefunden werden.
      \item Dies ist z.B. wichtig um zu bestimmen wieviel Wasserstoffbrücken im Mittel zu einem $O$-Atom in wässiriger Lösung gebildet werden.
  \end{numberlist}
\end{notebox}
\begin{sectionbox}[Beispiel Wasserstoffbrücken]\nospacing
  \hspace{-1em}\begin{figure}[H]	
    \centering{
      \def\svgwidth{100pt}
      \resizebox{0.6\linewidth}{!}{\input{figures/SMD/Wasserstoffbruecken.pdf_tex}}
    }
  \end{figure}
  1. Hügel/\rd{1. Koordinatenschale} entspricht den normalen zwei H-Atomen. Die weitern Extrema entsprechen weitern schwächeren H-Brücken.\\
  $\Rightarrow$ $\rd{g}(\rvec)\corresponds$ Anzahl an Wasserstoffbrücken.
  \begin{notebox}[Spezialfall]
    Bei $1000^{\circ}$ gibt es gar keine H-Brücken mehr.
  \end{notebox}
\end{sectionbox}
\begin{notebox}[Bemerkung]
\end{notebox}
\subsection{Thermodynamische Randbedingungen}
\begin{sectionbox}\nospacing
  Das gewählte Ensemble gibt konstante \imp{intensive} Grössen vor z.B. $\Tz,\pz,\muz,\ldots$, diese müssen von der MD-Simulation beachtet/beibehalten werden.\\
  Methoden um dies zu verwirklichen sind:
  \begin{circlelist}
      \item Constraint Methods
      \item Weak-Coupliong Methods (z.B. Berendson Thermostat)
      \item Extended System Methoden (z.B. extended-Lagrangian-Method)
      \item Stochastische Methoden (Lagrangian Dynamik)
  \end{circlelist}
\end{sectionbox}
	\vfill\columnbreak
\subsection*{Bsp. Thermostat: $T=$konst}
\begin{sectionbox}[\subsubsection{Constraint-Method}]\nospacing
  \imp{Idee}: Modifiziere die Netwon Bewegungsgleichung so, dass $\Tz=$konst.
  \begin{align}
    m\frac{\diff\vveci}{\diff t}=\ul{\frac{\diff\vec{p}_{\idxi}}{\diff t}}=\Fvec_{\idxi}-\underbrace{\rd{\mcc[\xi]\pveci}}_{\mathclap{\text{Zwangskraft}}}
  \end{align}
  \imp{Frage}: wie kann $\mcc[\xi]$ gewählt werden?
  \begin{align*}
    &\text{Wir wissen
      das:}&\Ez_{\text{kin}}^{\text{tot}}=\sum_{\idxi=1}^{\idxN}\frac{\pveci(t)^2)}{2m_{\idxi}}\eqs{\cref{eq:GasFreiheitsgrade}}\Nz\overbrace{\frac{3}{2}\kb\Tz(t)}^{\mathllap{\text{pro. Teilchen}}}
  \end{align*}
  Dies würde aber implizieren das $\Tz$ varien kann, was nicht sein darf.\\
  \imp{Idee}: sollten fordern das sich $\Tz$ nicht ändern darf.
  \begin{align}
    \Tz=\text{konst}&&\Longleftrightarrow&&\frac{\diff\Tz}{\diff t}=0
  \end{align}
  \begin{align*}
    \Longleftrightarrow 0&\eqs{!}\frac{\diff}{\diff
                          t}\left[\sum_{\idxi=1}^{\idxN}\frac{\pveci(t)^2)}{2m_{\idxi}}\right]
                          \eqs{\text{\rd{C.R.}}}\sum_{\idxi=1}^{\idxN}\frac{\pveci}{m_{\idxi}}\ul{\frac{\diff\pveci}{\diff t}}\nalign
    &=\sum_{\idxi=1}^{\idxN}\frac{\pveci}{m_{\idxi}}\Fvec_{\idxi}-\mcc[\xi]\sum_{\idxi=1}^{\idxN}\frac{\pveci}{m_{\idxi}}\rd{\pveci}=\sum_{\idxi=1}^{\idxN}\frac{\pveci}{m_{\idxi}}\Fvec_{\idxi}-\mcc[\xi]\sum_{\idxi=1}^{\idxN}\frac{\pveci^2}{m_{\idxi}}
  \end{align*}
  \begin{empheq}[box=\widefbox]{align}
    &\Rightarrow&\mcc[\xi](t)=\frac{\sum_{\idxi=1}^{\idxN}\vveci(t)\cdot\Fvec_{\idxi}(t)}{2\Ez_{\text{kin}}(t)}
  \end{empheq}
\end{sectionbox}
\begin{notebox}[Probleme]\nospacing
  \begin{numberlist}
      \item 
        $\Tz(t_0)$ kommt in diesem Verfahren nicht vor und $\Tz(t)=$konst. ist nicht notwendigerweise $\Tz(t)=\Tz(t_0)$\\
        $\Rightarrow$ nicht so leicht die Anfangstermperatur ein zu stellen.
      \item Einfaches scalen der Geschwindigkeit lässt keine Temperaturfluktuation \rd{einzlener} Teilchen zu, die im kanonischen Ensemble aber vorhanden sind:
        \begin{align*}
          \vveci^{\text{new}}=\lambdac\vveci^{\text{old}}
        \end{align*}
        \begin{align*}
          \lambdac=\sqrt{\frac{\Tz}{\Tz_0}}&&\text{mit}&&\obs{\Tz}\eqs{\cref{eq:GasFreiheitsgrade}}\frac{2}{3\Nz\kb\Ez_{\text{kin}}^{\text{tot}}} 
        \end{align*}
        $\Rightarrow$ Wenn wir nun jeden Zeitschritt die Temperatur anpassen, wird die kinetische Energie aller Teilchen einen konstanten Wert Annehmen, dies entspricht nicht dem
        kanonischen Ensemble.
  \end{numberlist}
\end{notebox}
\begin{sectionbox}[\subsubsection{Weak-Coupeling/Berendson-Thermostat}]\nospacing
  \rdb{Thermostate}/\rdb{Temperaturregler}: helfen dabei korrekte Proben von bestimmten Ensembles $(\Nz,\Vz,Tz),(\Nz,\pz,\Tz)$ zu erhalten.
  Dies verwicklichen sie durch Anpassung der $\Tz$ Temperatur des Systems in einer bestimmmten Art und Weise.\\
  Meist wird in MD-Simulatonen mittels Äquipartitionstheorem \cref{eq:Aquipartitionstheorem} die Temperatur $\Tz$, über die kinetische (gesamt) Energie des Systems
  berechnet.\\
  MD-Simulationen sind meistens relativ klein und weisen daher eher grössere Fluktuationen auf.\\
  Ohne Fluktationen der Temperatur $\Longleftrightarrow$ kin. Energie einzelner Teilchen können MD-Simulationen jedoch nicht die Korrekten Bewgungsgleichugen,
  die z.B. mit dem Kanonischen Ensemble übereinstimmen beschreiben.
  Das Ziel des Thermostat ist es daher dafür zu sorgen das:
  \begin{numberlist}
      \item Die korrekte gemittelte Temperatur des Systems konstant bleibt.
      \item Die Fluktuationen die korrekte Grösse aufweisen.
  \end{numberlist}
  Im Beredson Thermostat wird die Temperatur des Systems, über die Geschwindiglkeiten so skaliert, dass die Fluktuationen exponentiell mit einer Zeitkonstante
  $\lambdac$ abhnehmen.\\
  \imp{In anderen Worten}: versucht das Beredson Thermostat die Abweichung der aktuellen Temperatur $\Tz$ von der vorgegebenen $\Tz_0$ zu korregieren in dem es
  die Geschwindigkeiten mit einem Faktor $\lambdac$ multipliziert $\Longleftrightarrow$ Übertrangungsrate der Wärme, so das sich die Systemdynamik an $\Tz_0$ anpasst.
  \begin{align}
    \frac{\diff\Tz}{\diff t}=\frac{1}{\tauc}\left[\Tz_0-Tz(t)\right]&&\Longleftrightarrow&&\Tz=\Tz_0-\mcc\e^{-\frac{t}{\tauc}}\label{eq:berendsonFaktor}
  \end{align}
\end{sectionbox}
\begin{notebox}[Vorteil]
  Erlaubt Temperaturfluktuationen einzlner Teilchen und damit korrete Bewegungsgleichungen.
\end{notebox}
\begin{sectionbox}[Herleitung]\nospacing
  \begin{align*}
    &\imp{\text{Idee}:}&\vveci^{\text{new}}=\lambdac\vveci^{\text{old}},\qquad\forall\idxi=1,\ldots,\idxN
  \end{align*}
  \begin{align*}
    &\Rightarrow&\ul{\Delta\Ekin}=&\sum_{\idxi=1}^{N}\frac{1}{2}m_{\idxi}\lambdac^2\vveci^2(t)-\sum_{\idxi=1}^{N}\frac{1}{2}m_{\idxi}\vveci^2(t)\nalign
    &&=&(\lambdac^2-1)\sum_{\idxi=1}^{N}\frac{1}{2}m_{\idxi}\vveci^2(t)\nalign
    &&\eqs{\substack{\cref{eq:Aquipartitionstheorem}\\f=3}}&(\lambdac^2-1)\Nz\frac{3}{2}\kb\Tz(t)
  \end{align*}
  Für das kanonische Ensemble gilt aber auch:
  \begin{align*}
    \uldotted{\Delta\Ekin}\eqs{f=3}\Cv\Delta\Tz^{\text{tot}}=\Nz\underbrace{3\Cv^{\text{dof}}\Delta\Tz}_{\mathclap{\text{pro. Teilchen}}}
  \end{align*}
  \begin{notebox}[Nebenbemerkung]
    $\Tz$ ist die aktuelle Temperatur und $\Delta\Tz$ ist Temperaturänderung pro. Zeitschritt.
  \end{notebox}
  \begin{align*}
    \ul{\Delta\Ekin}=\uldotted{\Delta\Ekin}
  \end{align*}
  \begin{align*}
    &\Rightarrow& \Delta\Tz=&(\lambdac^2-1)\frac{1}{2}\frac{\kb\Tz(t)}{\Cv^{\text{dof}}}\nalign
    &&\lambdac^2=\ul[ulc2]{\frac{\Delta\Tz}{\Tz(t)}}\frac{2\Cv^{\text{dof}}}{\kb}+1
  \end{align*}
  \imp{Frage}: wie wählen wir nun $\lambdac$, so das $\Tz$ und $\Delta\Tz$ an die Zieltemperatur $\Tz_0$ angepasst werden?\\
  Mit der Diskretisierung von \cref{eq:berendsonFaktor} folgt:
  \begin{align*}
    \frac{\Delta\Tz}{\Delta t}=\frac{1}{\tauc}\left[\Tz_0-\Tz(t)\right]&&\Rightarrow&&\ul[ulc2]{\frac{\Delta\Tz}{\Tz(t)}}=\frac{\Delta t}{\tauc}\left[\frac{\Tz_0}{Tz(t)}-1\right]
  \end{align*}
  \begin{align*}
    \lambdac=\sqrt{\ul[ulc2]{\frac{\Delta t}{\tauc}\left[\frac{\Tz_0}{\Tz(t)}-1\right]}\frac{2\Cv^{\normalfont{d}}}{\kb}+1}\bua{\approx}{\normalfont{\rd{Taylor}}}\frac{\Delta t}{\tauc}\left[\frac{\Tz_0}{\Tz(t)}-1\right]\frac{2\Cv^{\text{d}}}{\kb}+1
  \end{align*}
\end{sectionbox}
\begin{sectionbox}[Einsetzen in die Bewegungsgleichung]\nospacing
  \begin{align*}
    m_{\idxi}\aveci=m_{\idxi}\frac{\diff (\lambdac\vveci)}{\diff t}\eqs{\text{P.R.}}m_{\idxi}\frac{\diff\lambdac}{\diff
    t}\vveci+m_{\idxi}\frac{\diff\vveci}{\diff t}\lambdac=\Fvec_{\idxi}\nalign
    \Rightarrow\quad m_{\idxi}\frac{\diff\vveci}{\diff t}=\frac{1}{\lambdac}\left[\Fvec_{\idxi}-m_{\idxi}\ul[ulc3]{\frac{\diff\lambdac}{\diff t}}\vveci\right]
  \end{align*}
  \begin{align*}
    &\imp{\text{Mit}: }&\ul[ulc3]{\frac{\diff\lambdac}{\diff t}}\approx\ul[ulc3]{\frac{\Delta\lambdac}{\Delta t}}=\frac{1}{\Tz(t)}=\frac{\Delta t}{\tauc}\left[\frac{\Tz_0}{Tz(t)}-1\right]
  \end{align*}
  \begin{align}
    &\text{Folgt dann}: &\boxed{m_{\idxi}\frac{\diff\vveci}{\diff t}=\Fvec_{\idxi}-m_{\idxi}\frac{\Cv^{\text{dof}}}{\kb}\left[\frac{\Tz_0}{\Tz(t)}-1\right]\vveci}
  \end{align}
\end{sectionbox}
\begin{notebox}[Probleme]
  \begin{numberlist}
      \item Auch der Berendson Thermostat kann die Fluktuationen nicht
    akkurat darstellen.
      \item Funktioniert schlecht in Systemen mit wenigen Freiheitsgraden z.B. zur Berechnung der freien Energie eines Systems mit Komponenten die im Glg. sehr
    wenig Freiheitsgrade besitzen (kleine nicht interagierende Moleküle).
  \end{numberlist}
\end{notebox}
\begin{sectionbox}[\subsubsection{Extended-Lagrangian-Method}]\nospacing
  \imp{Idee}: Regulierung der Temperatur durch Einführung eines zusätzlichen Freiheitsgrades.
  \begin{align*}
    \vveci:=\mca[s]\frac{\diff\rveci}{\diff t}&&\Rightarrow&&\text{Konjugierter Impuls: }\vec{p}_{\mca[s]}=m_{\idxi}\vveci
  \end{align*}
  \begin{align*}
    &\Rightarrow& \Ekin=\frac{1}{2}m_{\mca[s]}\left(\frac{\diff\mca[s]}{\diff t}\right)^2
  \end{align*}
  Ankoppelung an die Temperatur durch Wahl der pot. Energie:
  \begin{align*}
    \Wpot_{\mca[s]}=\underbrace{3N+1}_{\mathclap{\text{\# aller Dofs}}}\kb\Tz_0\cdot\overbrace{\ln(\mca[s])}^{\mathclap{\text{Wichtig für harm. Ensemble}}}
  \end{align*}
  \begin{flalign*}
    &\Rightarrow \text{\rd{Lagrange Fkt.}}:&\LT=\Ekin^{\text{tot}}-\Wpot^{\text{tot}}=
  \end{flalign*}
  \begin{align}
    &=\sum_{\idxi=1}^{N}\frac{1}{2}m_{\idxi}\vveci^2-\Wpot+\left[\frac{1}{2}m_{\mca[s]}\left(\frac{\diff\mca[s]}{\diff t}\right)^2-(3N-1)\kb\Tz_0\ln(\mca[s])\right]
  \end{align}
  \begin{figure}[H]
    \vspace{-3em}
    \centering
    \begin{tikzpicture}[node distance=1cm, every node/.style={fill=sectionbox, font=\sffamily}]
      \node (start)   {};
      \node (end)[rectangle, draw, below of=start]          {Euler-Lagrange Bewegungsgleichungen};
      % Draw edges
      \draw [line width=1pt,->] (start) -- node {\rd{Hamilton Prinzip}} (end);
    \end{tikzpicture}
  \end{figure}
\end{sectionbox}
\begin{sectionbox}[\subsubsection{Langerin-Method}]\nospacing
  \imp{Idee}: Benutze Langerin-Dynamik ($\corresponds$Stochastische Methode) $\rightarrow$ Einführung eines Reibungsterms:
  \begin{align}
    m_{\idxi}\frac{\diff\vveci}{\diff t}=\Fvec_{\idxi}-\rd{m_{\idxi}\gammac_{\idxi}\vveci+\Fvecst_{\idxi}}\Rightarrow\obs{\Tz}=\frac{\obs{\Fvec_{\idxi}^{2(st)}}}{6m_{\idxi}\gammac_{\idxi}\kb}\equiv\Tz_0
  \end{align}
\end{sectionbox}
\subsection*{Berechnung der Freien Energie}
\begin{sectionbox}[Vorgehen]\nospacing
  \imp{Eigentlich}: benötigt man $\Zz$ um daraus dann $\Spot$ und daraus dann $\Apot$ (oder auch $\Gpot$) zu berechnen.
  \begin{align*}
    \Apot&(\Nz,\Vz,\Tz)=\nalign
    &-\kb\Tz\ln \underbrace{\left[\frac{1}{h^{3N}N!}\iint\exp{\left\{-\frac{\Hamkl(\vec{p}^N,\rvec^N)}{\kb\Tz}\right\}\diff\rvec^N\diff\vec{p}^N}\right]}_{\Zz}
  \end{align*}
  Allerdings interessieten wir uns nur für die Änderung der freien Energie und nicht deren exakten Wert.
\end{sectionbox}
\begin{sectionbox}[Partielle Integration]\nospacing
  \begin{align}
    \Apot&(\Nz+1,\Vz,\Tz)-\Apot(\Nz,\Vz,\Tz)=\nalign
    &-\kb\Tz\ln\int\obs{\exp\left(-\frac{\Delta\Wpot}{\kb\Tz}\right)}_{\rvec_1,\ldots,\rvec_N}\diff^3\rvec^{N+1}
  \end{align}
  \begin{align*}
    &\text{Mit}:&\Delta\Wpot=\Wpot(\rvec_1,\ldots,\rvec_{N+1})-\Wpot(\rvec_1,\ldots,\rvec_N)
  \end{align*}
\end{sectionbox}
\begin{sectionbox}[Temperatur Integration]\nospacing
  \begin{align}
    &\frac{\Apot(\Tz_{\mcb})}{\Tz_{\mcb}}-\frac{\Apot(\Tz_{\mca})}{\Tz_{\mca}}=
    \int\limits_{\frac{1}{\Tz_{\mca}}}^{\frac{1}{\Tz_{\mcb}}}\underbrace{\obs{\Hamkl}_{\Tz}}_{=\Upot^{\text{phän.}}}\diff \frac{1}{\Tz}
  \end{align}
  \begin{align*}
    &\text{Mit}:&\Apot=\Upot-\Tz\Spot&&\Rightarrow&&\frac{\Apot}{\Tz}=\frac{\Upot}{\Tz}-\Spot
  \end{align*}
  \begin{align*}
    \obs{\Hamkl}=\Upot=\frac{\diff \frac{\Apot}{\Tz}}{\diff \Tz^{-1}}
  \end{align*}
\end{sectionbox}
\begin{sectionbox}[Druck Integration]\nospacing
  \begin{align}
    &\Apot(\Vz_{\mcb})-\Apot(\Vz_{\mca})=
      \int\limits_{\Vz_{\mca}}^{\Vz_{\mcb}}\pz\diff\Vz
  \end{align}
\end{sectionbox}
\begin{sectionbox}[Verallgemeinerte Integration]\nospacing
  \begin{align}
    &\Apot(\lambdac_{\mcb})-\Apot(\lambdac_{\mca})=
      \int\limits_{\lambdac_{\mca}}^{\lambdac_{\mcb}}\obs{\pfrac{\Hamkl(\lambdac)}{\lambdac}}_{\lambdac}\diff\lambdac
  \end{align}
\end{sectionbox}
%%% Local Variables:
%%% mode: latex
%%% TeX-master: "../formularySPCS"
%%% End:
