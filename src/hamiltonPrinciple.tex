\begin{defnbox}
  \begin{defn}[Wirkunsfunktional]
    \begin{equation*}
      \Sv[\vec{q}]:=\int_{t_1}^{t_2}\LT(\vec{q},\dot{\vec{q}},t)\diff
      t=\int_{t_1}^{t_2}\ul{\vec{K}(q,\dot{q},t)}-\ul[ulc2]{\vec{W}(q,\dot{q},t)}\diff t
    \end{equation*}
  \end{defn}
\end{defnbox}
\begin{principbox}
  \begin{princip}[Hamilton Prinzip]
    Die Wahre entwickelung $\vec{q}$ eines Systems beschrieben durch $N$ generalisierten Koordinaten $\vec{q}=(\vec{q}_1,\ldots,\vec{q}_N)$, im Konfigurationsraum $\R^f$, zwischen zwei Zuständen $\vec{q}(t_1)$ und $\vec{q}(t_2)$ is ein
    \rd{stationärer Punkt}(=Pkt. an dem die Variation Null ist).
    \begin{equation}
      \boxed{\delta\Sv\stackrel{!}{=}0}
    \end{equation}
  \end{princip}
\end{principbox}
\begin{notebox}[Bemerkung]
  Dies ist einläuchtend wenn man sich überlegt dass man für den Weg des geringsten Widerstandes eine möglichst kleine
  \ul{kinetische Energie} und grosse \ul[ulc2]{pot. Energie} möchte.
\end{notebox}
\begin{proofbox}\nospacing
   \begin{proof} Herleitung der E.L. Gl. mittels Hamiltonprinzip
     \begin{flalign*}
       &\text{\imp{Sei}:}
       &&\vec{y}_{\epsilon}(\vec{\var}):=\qv(t)+\epsilon\var(t)&\text{mit}&&
       \var\in C^{\infty}_0
     \end{flalign*}
     \begin{flalign*}
     &\text{\imp{Sei}:} &&\Phi_{\epsilon}:=\int_{t_1}^{t_2}\LT(\vec{y}(t),\dot{\vec{y}}(t),t)\diff t=\int_{t_1}^{t_2}\LT_{\epsilon}\diff t
     \end{flalign*}
     Wir wissen das $\Sv[\vec{x}]$ ein Minimum für $\vec{x}=\vec{q}$ hat $\Rightarrow$ $\Phi(\epsilon)$ muss ein Minimum
     für $\epsilon=0$ haben.
     \begin{align*}
       \Phi'(\epsilon)&=\frac{\diff\LT_{\epsilon}}{\diff\epsilon}=\frac{\diff}{\diff\epsilon}\int_{t_1}^{t_2}\LT_{\epsilon}\diff
       t=\int_{t_1}^{t_2}\frac{\diff\LT_{\epsilon}}{\diff\epsilon}\diff t\nalign
                 \frac{\diff\LT_{\epsilon}}{\diff\epsilon}&=\frac{\partial\LT_{\epsilon}}{\partial t}\frac{\diff t}{\diff\epsilon}+
                                                            \frac{\partial\LT_{\epsilon}}{\partial\vec{y}_{\epsilon}}\frac{\diff\vec{y}_{\epsilon}}{\diff\epsilon}+
                                                            \frac{\partial\LT_{\epsilon}}{\partial\dot{\vec{y}}_{\epsilon}}\frac{\diff\dot{\vec{y}}_{\epsilon}}{\diff\epsilon}\nalign
                                                            &=\frac{\partial\LT_{\epsilon}}{\partial\vec{y}_{\epsilon}}\var(t)+
                                                            \frac{\partial\LT_{\epsilon}}{\partial\dot{\vec{y}}_{\epsilon}}\dot{\var}(t)
     \end{align*}
     Wenn $\epsilon=0$ gilt $\vec{y}_{\epsilon}=\vec{q}(t)$ und $\LT_{\epsilon}=\LT(\vec{q},\dot{\vec{q}},t)$
     \begin{align*}
       \Phi'(0)&=\left.\frac{\diff\Phi}{\diff\epsilon}\right\vert_{\var=0}=
                 \int_{t_1}^{t_2}\left[\frac{\partial\LT}{\partial\vec{q}}\var(t)+
                                                            \frac{\partial\LT}{\partial\dot{\vec{q}}}\dot{\var}(t)\right]\nalign
               &\stackrel{\text{\rd{I.B.P.}}}{=}\int_{t_1}^{t_2}\left[\frac{\partial\LT}{\partial\vec{q}}-
                 \frac{\diff}{\diff t}\frac{\partial\LT}{\partial\dot{\vec{q}}}\right]\var(t)+\underbrace{\left.\left[\var(t)\frac{\partial\LT}{\partial\dot{\vec{q}}}\right]\right\vert^{t_2}_{t_1}}_{=0}\nalign
               0&\stackrel{\text{\rd{F.L.C.V}}}{=}\frac{\partial\LT}{\partial\vec{q}}-
                 \frac{\diff}{\diff t}\frac{\partial\LT}{\partial\dot{\vec{q}}}
     \end{align*}
   \end{proof} 
\end{proofbox}
%%% Local Variables:
%%% mode: latex
%%% TeX-master: "../formularySPCS"
%%% End:
