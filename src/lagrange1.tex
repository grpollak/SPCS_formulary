\begin{sectionbox} \nospacing
 Verallgemeinerung der Newtonschen Axiome zur Lösung von Problemen mit Zwangsbedingungen $\gz$.
 \begin{align}\label{eq:lagr1}
   m\ddot{\vec{r}}&=\vec{F}+\Zw & \text{\rd{Zwangskraft}:}\qquad \Zw
 \end{align}
\end{sectionbox}
  \todo[inline]{Fix unwanted stretching in Headers}
\begin{defnbox}\nospacing
  \begin{defn}[\\\hbox{Holonome/Integrable Zwangsbedinungen}]
    Sind Zwangsbedingungen die als Gleichungen zwischen den
    Ortsvariablen $\vec{r}_i$ des Systems und der Zeit in flogender Form formuliert werden können:
    \begin{align*}
      \gz_{\alphac}(\rv_1,\rv_2,\ldots,\rv_N,t)&=0 &\alphac=1,\ldots,R &&(R\text{-Zw.Bed.})
    \end{align*}
  \end{defn}
\end{defnbox}
\begin{defnbox}\nospacing
  \begin{defn}[\\\hbox{Anholonome Zwangsbedinungen}]
    Sind Zwangsbedingungen die nicht als holonome Zwangsbedingungen geschrieben werden können
    e.g. Ungleichungen oder:
    \begin{align*}
      \gz_{\alphac}(\rv_1,\rv_2,\ldots,\rv_N,\dot{\rv}_1,\dot{\rv}_2,\ldots,\dot{\rv}_N,t)&=0 &\alphac=1,\ldots,R 
    \end{align*}
  \end{defn}
\end{defnbox}
\begin{notebox}[Note: \normalfont{Zwangsbedingungen für ein Teilchen}]
  \begin{numberlist}
      \item Eine Zwangsbed. $\Rightarrow$ Einschränkung auf Fläche.
      \item Zwei unab. Zwangsbed. $\Rightarrow$ Einschränkung auf Kurve.
      \item Drei unab. Zwangsbed. $\Rightarrow$ alle drei Koordinaten $x,y,z$ sind festgelegt, keine Bewegung mehr erlaubt.
  \end{numberlist}
\end{notebox}
\begin{notebox}[Note: \normalfont{Zwangsbedungen für mehrere Teilchen}]
  \begin{numberlist}
    \item Die mögliche Anzahl $R$ der Bedingungen ist durch $R\leq3N-1$ begrenzt.
    \item Damit ist die Anzahl der Freiheitsgrade $f=3N-R$.
  \end{numberlist}
\end{notebox}
\begin{defnbox}
  \begin{defn}[Rheonome Zwangsbedingung]
    Sind zeitabhängige Zwangsbedingungen im Gegensatz dazu sind \rdb{skleronome Zwangsbedingungen} zeitunabhängig.
  \end{defn}
\end{defnbox}
\begin{sectionbox}[Zwangskräfte $\Zw$]\nospacing
  \imp{Problem}: wir kennen nur die Zwangsbedingungen $\gz$ sind nach \cref{eq:lagr1} aber an den Zwangskräften $\Zw$ interessiert.\\
  \imp{Idee}: Eine Einschränkung eines Teilchens durch eine holonome Zwb. $\gz$ beschränkt die Bewegung des Teilchens
  auf eine Fläche ein. Dies bedeutet aber wiederum das sich das Teilchen frei auf der Fläche bewegen kann und dies
  wiederum impliziert das die Kraft nur orthogonal wirken kann:
  \begin{align*}
		&\gz(\rv,t)=0 &\Longleftrightarrow&& \Zw\parallel\grad\gz(\rv,t) \\
    &\Rightarrow\text{\imp{Ansatz}:}&&& \Zw(\rv,t)=\lambdac(t)\grad\gz(\rv,t)
  \end{align*}
\end{sectionbox}
\begin{emphbox}[\rdb{Lagrangegleichungen 1. Art}]\nospacing
  \begin{align}
  &\ul{m_{\idxn}\ddot{\vec{x}}_{\idxn}=
  \vec{F}_{\idxn}+\sum_{\alphac=1}^R\lambdac_{\alphac}\frac{\partial\gz_{\alphac}(x_1,\ldots,x_{3N},t)}{\partial x_{\idxn}}}\label{eq:Lagrange1} \\
  &\ul[ulc2]{\gz_{\alphac}(x_1,\ldots,x_{3N},t)=0} \qquad \alphac=1,\ldots,R\quad \idxn=1,\ldots,3N \nonumber
  \end{align}
\end{emphbox}
\begin{notebox}[Bemerkunge]
  \begin{numberlist}
      \item $\ul{3N}+\ul[ulc2]{R}$ Gleichungen für $3N+R$ unbekannte Funktionen $x_{\idxn}(t)$ und $\lambdac_{\alphac}(t)$.
      \item $\ul{3N}$ dgl. 2. Ordnung.
      \item $\ul[ulc2]{R}$ algb. Gleichungen.
  \end{numberlist}
\end{notebox}
\begin{sectionbox}[Vorgehen]\nospacing
  \begin{numberlist}
    \item Formulierung der Zwangsbedingungen $\gz$ und Aufstellung der Lagrangegleichungen.
    \item Elimination der $\lambdac_{\alphac}$.
    \item Lösung der Bewegunngsgleichungen $\Rightarrow\vec{x}$ und bestimmung der Integrationskonstanten.
    \item Bestimmung der Zwangskräften.
  \end{numberlist}
\end{sectionbox}
%%% Local Variables:
%%% mode: latex
%%% TeX-master: "../formularySPCS"
%%% End:
