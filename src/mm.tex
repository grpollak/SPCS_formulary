\begin{sectionbox}[Ziel]\nospacing
  Bisher haben wir idealisierte Systeme betrachtet \cref{defn:idealSys}.\\
  Nun wollen wir jedoch Wechselwirkungen ($\corresponds$ pot. Energie) zwischen Teilchen zulassen.
  Hierbei müssen wir zwischen der Quantemechanik und der klassischen Molekülmechanik unterscheiden.
\end{sectionbox}
\begin{sectionbox}[Problem]\nospacing
  \boxed{\text{Zustands\rd{summe}}}\tc{ulc1}{$\longrightarrow$}\boxed{\text{TD-Funktionen}} Verbindung mit der QM nur möglich, da
  wir diskrete, abzählbare Energiezustände/Eigenwerte haben $\Rightarrow$ nur lösbar für einfache Modelsysteme.\\
  \imp{Problem}: In der klassische Mechanik für molekulare Objekte (\tc{section}{MM}) sind die Energien allerdings \rd{kontinuirlich}
  und damit nicht abzählbare $\Rightarrow$ \rd{Summe} über Zustände nicht einfach berechenbar.
  \begin{align*}
    \rdb{\text{Pot. Energie}}: \Wpot=
      \begin{cases}
        \tura{\text{QM}}{\normalfont{Zustandsenergie diskret durchnum.}}: \text{Bekannt für Elementarteilchen.}\\
        \bdra{\text{MM}}{\normalfont{Zustandsenergie kontinuirlich}}: \text{leicht berechenbar.}\\
      \end{cases}
  \end{align*}
\end{sectionbox}
\begin{sectionbox}[Vorgehensweise]\nospacing
 \begin{circlelist}
     \item Finde einen analytischen Ausdruck oder eine ``vernünftige'' Approximation für die pot. Energie $\Wpot$ in eine klassischen
   mech. Beschreibung.
     \item Bestimme $\Zz$ und draus TD-Funktionen.
 \end{circlelist}
\end{sectionbox}
\begin{principbox}
  \begin{princip}[Korrespondenzprinzip]\nospacing \leavevmode\\
    Für makroskopische Systeme muss die q.m. Beschreibung in die klassische Mechanik übergehen.
    \begin{align*}
      \lim_{\Nz\to\infty}\sum_{\nqz}\left(\text{Disk. q.m. Energie}\right)_{\nqz}&&\Longleftrightarrow&&\text{kont. Energie} 
    \end{align*}
  \end{princip}
\end{principbox}
\subsection{Berechnung von $\Wpot=0$}
\begin{sectionbox}[\subsubsection{Translation ($\corresponds$ Teilchen im 3D Kasten)}]\nospacing
  \imp{Betrachte} System aus $\Nz$ Teilchen \imp{ohne} W.W. ($\Wpot=0$).
  \begin{align*}
    &\text{\rd{Klassisch}:}& \Hamkl&=\Ez_{\text{kin.}}=\sum_{\idxi}^{N}\frac{\pvec_{\idxi}^2}{2m}+0&&\nalign
    &\text{\rd{QM}:}& \Ham&=\Ez_{kin.}(\nqz_x^{(\idxi)},\nqz_y^{(\idxi)},\nqz_z^{(\idxi)})+0\nalign
                     &&&=\sum_{\idxi}^{N}\frac{\pi^2\hpr^2}{2mL^2}\left(\nqz_x^{(\idxi)}+\nqz_y^{(\idxi)}+\nqz_z^{(\idxi)}\right)&\forall\idxi
  \end{align*}
\end{sectionbox}
\begin{sectionbox}[$\Zz_{\text{tr.}}$ Quantenmechanisch]\nospacing
  \begin{align*}
    \Zz_{\text{tr.}}^{\text{QM}}(\Nz,\Vz,\Tz)=\frac{\zz^{\Nz}_{\text{tr}}}{\Nz!}=\prod_{\idxi}^{\Nz}\frac{1}{\Nz!}\sum_{\nqz_x^{(\idxi)},\nqz_y^{(\idxi)},\nqz_z^{(\idxi)}}\e^{-\frac{1}{\kb\Tz}\Ez_{\text{kin.}}^{\text{QM}}}
  \end{align*}
  Da L eine Makroskopische Grösse ist, kann man die Summe im Limes durch ein Integral erstezen:
  \begin{align*}
    \sum_{\nqz_x}\e^{-\betac\frac{\pi^2\hpr^2}{2mL^2}\nqz_x^2}\xrightarrow[\text{\rd{K.P.}}]{\lim\limits_{\Nz\to\infty}\nqz_x\to x}
    \int_{0}^{\infty}\e^{-\betac\frac{\pi^2\hpr^2}{2mL^2}x^2}\diff x
  \end{align*}
  \begin{align*}
     &\text{Mit }\int_{0}^{\infty}\e^{-\betac\frac{\pi^2\hpr^2}{2mL^2}x^2}\diff x\turla{=}{\cref{eq:exa2}}\frac{1}{2}\left(\frac{2 mL^2}{\betac\pi\hpr^2}\right)^{\frac{1}{2}}=\left(\frac{2 mL^2}{4\betac\pi\hpr^2}\right)^{\frac{1}{2}}
  \end{align*}
  \begin{align*}
    \zztrans=\zztrans_{x}\cdot\zztrans_{y}\cdot\zztrans_{z}=\left(\frac{2 mL^2}{4\betac\pi\hpr^2}\right)^{\frac{1}{2}+\frac{1}{2}+\frac{1}{2}}
  \end{align*}
  \begin{align}
    \boxed{\zztrans=\left(\frac{mL^2}{2\betac\pi\hpr^2}\right)^{\frac{3}{2}}=L^3\cdot\left(\frac{m}{2\betac\pi\hpr^2}\right)^{\frac{3}{2}}
    =\ul[ulc2]{V\cdot\left(\frac{m}{2\betac\pi\hpr^2}\right)^{\frac{3}{2}}}}
  \end{align}
\end{sectionbox}
\begin{sectionbox}[$\Zz_{\text{tr.}}$ KLassisch]\nospacing
  \begin{align*}
    &\text{\imp{Ansatz}: }&\e^{-\betac\Hamkl_{\text{kl.}}}=\exp\left[-\betac \left(\frac{\pimp_x^2+\pimp_y^2+\pimp_x^2}{2m}\right)\right]
  \end{align*}
  \begin{align*}
    &\zztrans_{\text{kl.}}=\mcc\underbrace{\left(\int_{-\infty}^{\infty}\diff^{3\Nz}r\right)}_{\Vz^{\Nz}}\cdot\int_{-\infty}^{\infty}\e^{-\betac \left(\frac{\pimp_x^{(i)2}+\pimp_y^{(i)2}+\pimp_z^{(i)2}}{2m}\right)}\diff^{3N}\pimp\nalign
                         &=\mcc\underbrace{\left(\int_{-\infty}^{\infty}\diff^{3\Nz}r\right)}_{\Vz^{\Nz}}\cdot\int_{-\infty}^{\infty}\exp\left[-\betac \left(\frac{\pimp_x^{2}+\pimp_y^{2}+\pimp_z^{2}}{2m}\right)\right]^{\Nz}\diff^{3N}\pimp\nalign
    &\eqs{\cref{eq:exa2}}\ul{\mcc\Vz^{\Nz}\left[2\pi m\kb\Tz\right]^{\frac{3\Nz}{2}}}
  \end{align*}
  \begin{align*}
    \ul{\zztrans_{\text{kl.}}}\eqs{\text{\rd{K.P.}}}\ul[ulc2]{\zztrans_{\text{QM}}}&&\Rightarrow&&\mcc=\frac{1}{\Nz!}\frac{1}{\hp^{3\Nz}}
  \end{align*}
  \begin{align}
    \boxed{\zztrans_{\text{kontin.}}=\frac{\Vz^{\Nz}}{\Nz!\hp^{3\Nz}}\left[2\pi m\kb\Tz\right]^{\frac{3\Nz}{2}}}
  \end{align}
\end{sectionbox}
\begin{sectionbox}[\subsubsection{Rotation ($\corresponds$ starrer Rotator)}]\nospacing
  \begin{flalign*}
    &\text{\imp{Wiederhohlung}: }&\zzrot&=\sum_{\Jqz}^{\infty}\left[(2\Jqz+1)\e^{-\betac\Ez_{\text{rot}}^{\Jqz}}\right]\nalign
    &&\Ez_{\text{rot}}^{\Jqz}&=\Jqz(\Jqz+1)\frac{\hpr^2}{2I}
  \end{flalign*}
  \imp{Annahme}: für ein makroskopisches System kann die Summe als kontinuierlich betrachtet werden.\\
  Mit Hilfe der Substitution $u:=\Jqz(\Jqz+1)=\Jqz^2+\Jqz$ folgt dann:\\
  \[\Rightarrow \frac{\diff u}{\diff \Jqz}=2\Jqz+1 \Rightarrow \diff\Jqz=\frac{\diff u}{2\Jqz+1}\]
  \begin{align*}
    \zzrot_{\text{kontin.}}&\approx\int_{0}^{\infty}(2\Jqz+1)\e^{-\frac{\betac\hpr^2\Jqz(\Jqz+1)}{2I}}\diff\Jqz=\int_{0}^{\infty}\e^{-\frac{\betac\hpr^2 u)}{2I}}\diff u\nalign
    &\eqs{\cref{eq:exa2}}\left.\left[-\frac{2I}{\betac\hpr^2}\e^{-\frac{\beta\hpr^2u}{2I}}\right]\right\rvert_{0}^{\infty}=\left(0-\left(-\frac{2I}{\betac\hpr^2}\right)\right)
  \end{align*}
  \begin{align}
    \boxed{\zzrot_{\text{kontin.}}=\frac{2I}{\betac\hpr^2}}
  \end{align}
\end{sectionbox}
  \todo[inline]{Add right tikz Picture}
% \begin{sectionbox}[Recap]\nospacing
%    \begin{tikzpicture}[node distance=1cm,
%       every node/.style={fill=sectionbox, font=\sffamily}]
%     \node (TD)[rectangle, draw]          {Jede TD-Funktion};
%     \node (Zzt)[rectangle, draw, below of=TD]          {$\Zz$};
%     \node (Zz)[rectangle, draw, below of=Zzt]          {$\Zz=\frac{\zz^N}{N!}$};
%     \node (zz)[rectangle, draw, below of=Zz]             {$\zz_{\text{m}}=\zztrans\cdot\zzrot\cdot\zzvib\cdot\zzelek$};
%     % Draw edges
%     \draw [ulc1,->] (TD) -- node {berechenbar durch} (Zzt);
%     \draw [ulc2,->] (Zzt) -- node {Zerlegung} (Zz);
%     \draw [ulc3,->] (Zz) -- node {Zerlegung} (zz);
%   \end{tikzpicture}
%\end{sectionbox}
\subsection{Berechnung von $\Wpot\neq0$}
\begin{sectionbox}\nospacing
  Für $\Wpot\neq0$ ergibt sich für die kanonische Zustands\rd{funktion}:
  \begin{align*}
    &\Zzk=\frac{1}{\Nz!\hp^{3\Nz}}\int_{-\infty}^{\infty}\int_{-\infty}^{\infty}\e^{-\betac\Ham(\vec{r}^{\Nz},\vec{p}^{\Nz})}\diff^{3\Nz}\vec{r}\diff^{3\Nz}\vec{p}
  \end{align*}
  \begin{align*}
    \Ham(\vec{r}^{\Nz},\vec{p}^{\Nz})=\Ez_{\text{kin}}(\vec{p}^{\Nz})+\Wpot(\vec{r}^{\Nz})
  \end{align*}
  \begin{flalign*}
  &\text{\imp{Mit}}&&\text{Ortskoordinaten: }&\vec{r}^{\Nz}=(r_1,\ldots,r_{\Nz})\nalign
          &&&\text{Impulskoordinaten: }& \vec{p}^{\Nz}=(p_1,\ldots,p_{\Nz})
  \end{flalign*}
  \imp{Frage}: wie kann/soll man $\Wpot(\vec{r}^{\Nz})$ wählen?\\
\end{sectionbox}
\begin{sectionbox}[\imp{Ansatz}: Zerlegung+Parameterisierung]
  \begin{numberlist}
      \item \tc{section}{Zerlegung}: Abhängig davon was man beschreiben möchte: Atome, Atomgruppen $\corresponds$ ``\rd{united Atom}''
      (z.B. Methylgruppen) = \rd{Coarse grained}.\\
      $\Rightarrow$ bestimmt durch ``Teilchen im System''
      \item \textcolor{section}{Parameterisierung}: Empirisch oder aus $\Ez_{\text{el.}}$.
  \end{numberlist}
\end{sectionbox}
\begin{sectionbox}\nospacing
  \imp{Idee}: Moleküle werden durch Stäbe, Kugeln, Scheiben,\ldots representiert.\\
  \imp{Frage}: was sind die Freiheitsgrade dieser Entitäten?
  \begin{align*}
    \Wpot=\text{Bindende W.W.}+\text{Nicht bindende W.W.}
  \end{align*}
\end{sectionbox}
\begin{sectionbox}[\subsubsection{Bindende/Intramolekulare W.W.}]\nospacing
  \begin{figure}[H]	
    \centering{\vspace{-1em}\hspace{-1cm}
      \def\svgwidth{140pt}
      \resizebox{0.8\linewidth}{!}{\input{figures/MM/intereWW.pdf_tex}}
    }
  \end{figure}
  \vspace{-1em}
  Für diese intramolekularen Freiheitsgrade ist es üblich folgende Beiträge für $\Wpot$ zu definieren:
\end{sectionbox}
\begin{sectionbox}[Streckschwingung (harm. Parabelpotential)]\nospacing
  \begin{align*}
    &\Wpot^{\text{Bond}}(\vec{r}^N)=\sum_{\idxi}\frac{1}{2}k_{\idxi}^{\text{bond}}\left[b_{\idxi}(\rvec^N)-b_{\idxi}^{\circ}\right]& \forall\idxi\in\text{Bonds}
  \end{align*}
  \begin{flalign*}
        &\text{\imp{Mit}:} &&k_{\idxi}^{\text{bond}}: &&\text{Kraftkonstante}\nalign
        &&&b_{\idxi}(\rvec^N): &&\text{Abstand von zwei Atomen zu denen eine}\nalign
        &&&&&\text{chem. Bindung postuliert wird}\nalign
        &&&b_{\idxi}^{\circ}:&&\text{Glg. Bindungslänge der $\idxi$-ten Bindung}\nalign
        &&&&&\Longleftrightarrow\text{Minimums $\Ez_{el.}$}
  \end{flalign*}
\end{sectionbox}
\begin{sectionbox}[Biegeschwingung (harm. Parabelpotential)]\nospacing
  \begin{align*}
    &\Wpot^{\text{angle}}(\vec{r}^N)=\sum_{\idxj}\frac{1}{2}k_{\idxj}^{\text{angle}}\left[\theta_{\idxj}(\rvec^N)-\theta_{\idxj}^{\circ}\right]& \forall\idxj\in\text{Angle}
  \end{align*}
  \begin{flalign*}
        &\text{\imp{Mit}:} &&k_{\idxj}^{\text{angle}}: &&\text{Kraftkonstante}\nalign
        &&&\theta_{\idxj}(\rvec^N): &&\text{Bindungswinkel von zwei Atomen zu}\nalign
        &&&&&\text{denen eine chem. Bind. postuliert wird}\nalign
        &&&\theta_{\idxj}^{\circ}:&&\text{Glg. Bindungswinkel der $\idxj$-ten Bindung}\nalign
        &&&&&\Longleftrightarrow\text{Minimums $\Ez_{el.}$}
  \end{flalign*}
\end{sectionbox}
\begin{sectionbox}[Torsion]\nospacing
  \begin{align*}
    &\Wpot^{\text{torsion}}(\vec{r}^N)=\sum_{\idxk}\frac{1}{2}k_{\idxk}^{\text{torsion}}\left\{1+\cos[m_{\idxk}\Phi_{\idxk}(\rvec)-\delta_{\idxk}]\right\}\nalign
    &\forall\idxk\in\text{Diederwinkel}\qquad\text{\imp{Mit}:}\quad \Phi_{\idxk} \text{Diederwinkel}
  \end{align*}
\end{sectionbox}
\begin{notebox}[Fitparameter]\nospacing
  \begin{align*}
    k_{\idxi}^{\text{bond}},k_{\idxi}^{\text{bond}},m_{\idxk},\delta_{\idxk}
  \end{align*}
\end{notebox}
\begin{sectionbox}[\subsubsection{Nicht bindende Wechselwirkungen}]\nospacing
  \begin{figure}[H]	
    \centering{
      \def\svgwidth{140pt}
      \resizebox{0.7\linewidth}{!}{\input{figures/MM/vdw.pdf_tex}}
    }
    \vspace{-1em}
  \end{figure}
  Für diese intermolekularen Freiheitsgrade ist es üblich folgende Beiträge für $\Wpot$ zu definieren:
\end{sectionbox}
\begin{sectionbox}[Coulomb W.W.]\nospacing
  \begin{align*}
    &\Wpot^{\text{Coulomb}}(\vec{r}^N)=\frac{1}{4\pi\epsilon_{0}\epsilon_{r}}\sum_{\substack{\text{Paare}\\\idxi<\idxj}}\frac{q_{\idxi}q_{\idxj}}{\rvec_{\idxi\idxj}}
  \end{align*}
\end{sectionbox}
\begin{sectionbox}[Van der Waals W.W. (\rd{6-12-/Lenard-Jones Potential})]\nospacing
  \begin{align*}
    &\Wpot^{\text{VdW}}(\vec{r}^N)=\sum_{\substack{\text{Paare}\\\idxi<\idxj}}4\epsilon_{\idxi\idxj}\left[\left(\frac{\sigma_{\idxi\idxj}}{\rvec{\idxi\idxj}}\right)^{12}-\left(\frac{\sigma_{\idxi\idxj}}{\rvec{\idxi\idxj}}\right)^6\right]
  \end{align*}
  \begin{flalign*}
        &\text{\imp{Mit}:} &&\epsilon_{\idxi\idxj}: &&\text{Potentialtopftiefe}\nalign
        &&&\text{$\left(\frac{\sigma_{\idxi\idxj}}{\rvec{\idxi\idxj}}\right)^{12}$}: &&\text{Abstossend \& kurzreichweitige Energie}\nalign
        &&&&&\text{$\Rightarrow$ Dominiert bei kurzer Reichweite}\nalign
        &&&\text{$\left(\frac{\sigma_{\idxi\idxj}}{\rvec{\idxi\idxj}}\right)^{6}:$}&&\text{Anziehend \& langreichweitige Energie}\nalign
        &&&&&\text{$\Rightarrow$ Dominiert bei langer Reichweite}
  \end{flalign*}
  \begin{figure}[H]
    \centering
    \begin{subfigure}{.45\columnwidth}
      \centering{
        \def\svgwidth{100pt}
        \resizebox{\linewidth}{!}{\input{figures/MM/potTopf.pdf_tex}}
      }
      \vspace{-1.5em}
      \caption{}
      \label{fig:}
    \end{subfigure}%
    \hfil
    \begin{subfigure}{.45\columnwidth}
      \centering{
        \def\svgwidth{100pt}
        \resizebox{\linewidth}{!}{\input{figures/MM/LenardJonesPotential.pdf_tex}}
      }
      \vspace{-1.5em}
      \caption{}
      \label{fig:}
    \end{subfigure}
  \end{figure}
\end{sectionbox}
\begin{sectionbox}[$R_{\text{min}}$]\nospacing
  \begin{align*}
    \vec{F}(\vec{r})&=-\nabla\Wpot(\rvec)=\frac{\diff}{\diff\vec{r}}4\epsilon\left[\rd{-}\sigma^{12}\rvec^{-12}\rd{+}\sigma^6\rvec^{-6}\right]\nalign
    &=4\epsilon\left[12\sigma^{12}\rvec^{-13}-6\sigma^6\rvec^{-7}\right]\eqs{!}0\iff \rvec^6=2\sigma^6
  \end{align*}
  \begin{align}
    R_{\text{min}}=\sqrt[6]{2}\sigma
  \end{align}
\end{sectionbox}
\begin{notebox}[Fitparameter]\nospacing
  \begin{align*}
    q_{\idxi},q_{\idxj},\epsilon_{\idxi\idxj},\sigma_{\idxi\idxj}
  \end{align*}
\end{notebox}
\begin{notebox}[Nebenbemerkung]\nospacing
  \begin{numberlist}
    \item \rd{Paarpotentialnäherung}: Mehrkörperwechselwirkung wurde vernachlässigt.
    \item Die Kräfte auf alle ``Teilchen'' des Systems erhählt man als Gradienten der pot. Energie
      \begin{align*}
        \vec{F}_{\idxi}=-\nabla_{\idxi}\Wpot(\rvec^N)=\vect{\pfrac{}{x_{\idxi}} & \pfrac{}{y_{\idxi}}& \pfrac{}{z_{\idxi}}}^{T}
      \end{align*}
  \end{numberlist}
\end{notebox}
\begin{notebox}[Kraft die von Atom j verursacht wird und auf Atom i wirkt]\nospacing
  \begin{align*}
    f_{x_{\idxi}}=&-\sum_{\substack{\idxi\\\idxi\neq\idxj}}^{N}\frac{\partial\Wpot(\rvec_{\idxi},\rvec_{\idxj})}{\partial\rvec_{\idxi\idxj}}\cdot\pfrac{r_{\idxi\idxj}}{x_{\idxi}}\nalign
    \eqs{\text{Ein Nachbaratom}}=&-\frac{\partial\Wpot(\rvec_{\idxi},\rvec_{\idxj})}{\partial\rvec_{\idxi\idxj}}\cdot\pfrac{r_{\idxi\idxj}}{x_{\idxi}}
  \end{align*}
  \begin{align*}
    \frac{\partial\Wpot(\rvec_{\idxi},\rvec_{\idxj})}{\partial\rvec_{\idxi\idxj}}=&
    4\epsilon_{\idxi\idxj}\left[-12\left(\frac{\sigma_{\idxi\idxj}}{\rvec{\idxi\idxj}}\right)^{12}+6\left(\frac{\sigma_{\idxi\idxj}}{\rvec{\idxi\idxj}}\right)^6\right]\frac{1}{\rvec_{\idxi\idxj}}
  \end{align*}
  \begin{align*}
  &\pfrac{r_{\idxi\idxj}}{x_{\idxi}}=\frac{\partial}{\partial x_{\idxi}}\left[(x_{\idxi}-x_{\idxj})^2+(y_{\idxi}-y_{\idxj})^2+(z_{\idxi}-z_{\idxj})^2\right]^{\frac{1}{2}}\nalign
  &=\frac{1}{2}\left[(x_{\idxi}-x_{\idxj})^2+(y_{\idxi}-y_{\idxj})^2+(z_{\idxi}-z_{\idxj})^2\right]^{-\frac{1}{2}}2(x_{\idxi}-x_{\idxj})\nalign
    &=\frac{x_{\idxi\idxj}}{\rvec_{\idxi\idxj}}
  \end{align*}
  \begin{align*}
    f_{x_{\idxi}}=-4\epsilon_{\idxi\idxj}\left[-12\left(\frac{\sigma_{\idxi\idxj}}{\rvec{\idxi\idxj}}\right)^{12}+6\left(\frac{\sigma_{\idxi\idxj}}{\rvec{\idxi\idxj}}\right)^6\right]\frac{x_{\idxi\idxj}}{\rvec_{\idxi\idxj}^2}
  \end{align*}
\end{notebox}
\begin{sectionbox}[Damit ergibt sich]\nospacing
  \begin{align}
    \Wpot=\Wpot^{\text{bond}}+\Wpot^{\text{angle}}+\Wpot^{\text{torision}}+\Wpot^{\text{Coul.}}+\Wpot^{\text{VdW}}
  \end{align}
\end{sectionbox}
\begin{sectionbox}[\subsection{Berechnung der Zustandsfunktion}]\nospacing
  Wir haben nun einen analytischen ausdruck für $\Wpot$ gefunden und können nun die komplette Zustands\rd{funktion} berechnen:
  \begin{align*}
    &\Zzk=\frac{1}{\Nz!\hp^{3\Nz}}\iint_{-\infty}^{\infty}\e^{-\betac\Ham(\vec{r}^{\Nz},\vec{p}^{\Nz})}\diff^{3\Nz}\vec{r}\diff^{3\Nz}\vec{p}
  \end{align*}
  \begin{align*}
    \Ham(\vec{r}^{\Nz},\vec{p}^{\Nz})=\sum_{\idxi=1}^{N}\frac{\prescript{N}{}{\pvec}_{\idxi}^2}{2m_{\idxi}}+\Wpot(\vec{r}^{\Nz})
  \end{align*}
  \begin{alignat*}{3}
    \Zzk&\eqs{\Wpot\neq0}&&\mcc[\frac{1}{\hp^{3N}N!}]\int_{-\infty}^{\infty}\exp\left(-\betac\sum_{\idxi=1}^{N}\frac{\prescript{N}{}{\pvec}_{\idxi}^2}{2m_{\idxi}}\right)\diff^{3N}\pimp\nalign
    &&&\cdot\int_{-\infty}^{\infty}\e^{-\betac\Wpot(\rvec^N)}\diff^{3\Nz}r
  \end{alignat*}
  \begin{alignat*}{2}
    &\eqs{\cref{eq:eIntegral}}&\mcc[\frac{1}{\hp^{3N}N!}]\prod_{\idxi}^N \left(2\pi\m_{\idxi}\kb\Tz\right)^{\frac{3}{2}}\cdot\int\limits_{-\infty}^{\infty}\e^{-\betac\Wpot(\rvec^N)}\diff^{3\Nz}r
  \end{alignat*}
  Mit der Annahnme gleicher Massen: $m_{\idxi}=m\quad\forall\idxi$ folgt:
  \begin{align*}
        &\eqs{\hphantom{\cref{eq:eIntegral}}}\underbrace{\frac{1}{N!}\left(2\pi\hp^{-2}\m\kb\Tz\right)^{\frac{3N}{2}}}_{\equiv\tc{ulc1}{f}}\cdot\underbrace{\int\limits_{-\infty}^{\infty}\e^{-\betac\Wpot(\rvec^N)}\diff^{3\Nz}r}_{\equiv\Qzk}
  \end{align*}
  Damit ergibt sich für Observablen $\Omega$:
  \todo[inline]{Warum?}
  \begin{alignat*}{2}
    &\obs{\Omega}_{\Nz,\Vz,\Tz}&&=\frac{\iint_{-\infty}^{\infty}\Omega(\rvec^N)\e^{-\betac\Ham(\vec{r}^{\Nz},\vec{p}^{\Nz})}\diff^{3\Nz}\vec{r}\diff^{3\Nz}\vec{p}}{\Zzk}\nalign
    &&&=\frac{\int_{-\infty}^{\infty}\Omega(\rvec^N)\cdot\e^{-\betac\Wpot(\rvec^N)}\diff^{3\Nz}r\cdot\cancel{\tc{ulc1}{f}(\pvec^N)}}{\int_{-\infty}^{\infty}\e^{-\betac\Wpot(\rvec^N)}\diff^{3\Nz}r\cdot\cancel{\tc{ulc1}{f}(\pvec^N)}}\nalign
    &&&\equiv\int_{-\infty}^{\infty}\Omega(\rvec^N)\pbr(\rvec^N)\diff^{3\Nz}r
  \end{alignat*}
\end{sectionbox}
\begin{defnbox}\nospacing
  \begin{defn}[Konfigurationszustandssumme/integral]
    \begin{align}
      \Qzk=\int\limits_{-\infty}^{\infty}\e^{-\betac\Wpot(\rvec^N)}\diff^{3\Nz}r
    \end{align}
    \todo[inline]{Haben wir dann nicht ein Doppelintegral über $\Qzk$ wenn wir in $\pbr$ einsetzen?}
    \begin{flalign}
      &\text{\imp{Mit}}:&&\pbr(\rvec^N)=\frac{\e^{-\betac\Wpot(\rvec^N)}}{\Qzk}&\nalign
      &\text{und}:&&\obs{\Omega}_{\Nz,\Vz,\Tz}=\int\limits_{-\infty}^{\infty}\Omega(\rvec^N)\pbr(\rvec^N)\diff^{3\Nz}r
    \end{flalign}
  \end{defn}
\end{defnbox}
\begin{sectionbox}[\subsubsection{Berechnung von $\Qzk$}]\nospacing
  \imp{Problem}: Analytische Lösung i.d.R. nicht möglich.\\
  $\Rightarrow$ Nummerische Integration.\\
  \imp{Problem}: Bei 10 Gitterpunkten pro Freiheitsgrad  ergeben sich $10^{3\idxN}$ Punkte an denen der Integrand zu berechnen ist.
  Mit $\idxN=10^{24}$ Teilchen schon nicht mehr möglich.
\end{sectionbox}

	\vfill\columnbreak
\subsection{Stochastische Lösungen/Samplingsmethoden}
\begin{sectionbox}[\subsubsection{Random Sampeling}]\nospacing
  \begin{align*}
    \Qzk&=\int_{\rvec_{\mca}}^{\rvec_{\mcb}}\e^{-\betac\Wpot(\rvec^N)}\diff\rvec^N\qquad \rvec=(r_x,r_y,r_z)\nalign
    &=\obs{\e^{-\betac\Wpot(\rvec^N)}}\underbrace{\prod_{\idxi}^{3N}(\rvec_{\idxi,\mcb}-\rvec_{\idxi,\mca})}_{\text{Volumen der Domain}}\nalign
    &\approx\frac{1}{n_{\pb}}\sum_{\idxj=1}^{n_{\pb}}\prod_{\idxi=1}^{3N}(\rvec_{\idxi,\mcb}-\rvec_{\idxi,\mca})\cdot\e^{-\betac\Wpot(\rvec_{\idxi}^N)}
  \end{align*}
  \begin{wrapfigure}{l}{0.4\linewidth}	
    \centering{
      \vspace{-1em}
      \inkscape[80pt]{figures/MM/randomnSampeling.pdf_tex}
    }
  \end{wrapfigure}
  \begin{flalign*}
    &\hspace{-2em}\approx\frac{1}{n_{\pb}}\sum_{\idxj=1}^{n_{\pb}}\prod_{\idxi=1}^{3N}(\rvec_{\idxi,\mcb}-\rvec_{\idxi,\mca})\cdot\e^{-\betac\Wpot(\rvec_{\idxi}^N)}
  \end{flalign*}
  \begin{flalign*}
    &n_{\pb}:&& \text{Anzahl der gewählten}&\nalign
    &&&\text{Konfigurationen}&
  \end{flalign*}
\end{sectionbox}
\todo[inline]{Check index of $\Wpot: \idxi$ oder $\idxj$ und wie kommt man auf die Formel?}
\begin{notebox}[Note]
  \begin{numberlist}
      \item Die $n_{\pb}$ Konfigurationen im Intervall $[\rvec_{\mca}^N,\rvec_{\mcb}^N]$ werden zufäfllig gewählt.
      \item Alle $\rvec_{\idxi}^N$ haben diesielbe Wahrscheinlichkeit.
      \item Sehr ineffizient, da $n_{\pb}$ sehr gross sein muss.
  \end{numberlist}
\end{notebox}
\begin{sectionbox}[\subsubsection{Importance Sampeling}]\nospacing
  \imp{Idee}: beprobe/sample nicht zufällig sondern wähle die $\{\rvec_{\idxi}^N\}$ anhand einer Wahrscheinlichkeitsverteilung, so dass der \rd{Boltzmann Faktor} $\e^{-\betac\Wpot(\rvec_{\idxi}^N)}$ gross wird.
\end{sectionbox}
\begin{sectionbox}[\subsubsection{Metropolis-Monte-Carlo Algorithm}\normalfont{(Variante des Importance Sampling})]\nospacing
  \imp{Grundidee}:
  \begin{circlelist}
    \item Zufällige Auswahl an Konfigurationen $r_{\idxi}^N$.
    \item Übernehme oder verwerfe die Konfigurationen $r_{\idxi}^N$ auf Grund der Boltzmann Verteilung.
    \item Berechne $\obs{\Omega}_{\Nz,\Vz,\Tz}$ aus den akzeptierten $r_{\idxi}^N$.
  \end{circlelist}
\end{sectionbox}
\begin{sectionbox}[Konkrete Implementation]\nospacing
  \imp{Voraussetzung}: grosse Anzahl an Schritten.
  \begin{circlelist}
    \item Generiere eine neue Konfiguration: $\rvec_{\idxi+1}^N=\rvec_{\idxi}^N+\Delta\rvec^N$\\
     $\Delta\rvec^N$: gibt verschiedene Möglichkeiten dies zu wählen.
    \item Berechne die Änderung der Energie\\\ctr{$\Delta\Wpot=W(\rvec_{\idxi+1}^N)-W(\rvec_{\idxi}^N)$}
    \item Akzeptiere $\rvec_{\idxi+1}^N$ auf Grund der Boltzmann-Verteilung falls:
    \begin{numberlist}
        \item $\Delta\Wpot\leq0$ (pot. Energie nimmt ab)\hfil\imp{oder}
        \item $\Delta\Wpot>0$\hfil \imp{und} \hfil$\e^{-\frac{\Delta\Wpot}{\kb\Tz}}>\epsilon$\\
          \imp{Mit}\hfil der Zufallszahl: $\epsilon\in[0,1]$
    \end{numberlist}
    ansonsten verwerfe $\rvec_{\idxi+1}^N$.
    \item Repeat.
  \end{circlelist}
  \begin{figure}[H]	
    \centering{
      \def\svgwidth{160pt}
      \resizebox{0.7\linewidth}{!}{\input{figures/MM/MMC.pdf_tex}}
    }
  \end{figure}
\end{sectionbox}
\begin{notebox}[Parameter des MC-Algorithmus]
  \begin{numberlist}
      \item Anzahl der Schritte.
      \item $\Delta\rvec^n$.
  \end{numberlist}
\end{notebox}
\begin{notebox}[Effizienz der Beprobung]\nospacing
  \begin{numberlist}
      \item Schrittgrösse $\Delta\rvec$ sehr/zu gross $\Rightarrow$ $\Delta\Wpot$ gross:
      \begin{align*}
        \Rightarrow \e^{-\frac{\Delta\Wpot}{\kb\Tz}}\approx0&&\Rightarrow&&\e^{-\frac{\Delta\Wpot}{\kb\Tz}}\stackrel{\text{i.d.R.}}{\ngtr}\epsilon
      \end{align*}
      $\Rightarrow$ verwerfen für $\Delta\Wpot>0$ so gut wie immer.
        \item Schrittgrösse $\Delta\rvec$ zu klein:\\
      Limitierte Beprobung, suchen nur kleinen Teil des Raumes ab.
  \end{numberlist}
  $\Rightarrow$ Effizienz hängt von der Balance zwischen der Schrittgröse und der Akkzeptanz/Verwerfungsrate ab.
\end{notebox}
\begin{notebox}[Nebenbemerkung]
  Wir benutzen hier die Boltzmann-Gewichtung und beproben das kanonische Ensemble.\\
  $\Rightarrow$ falls man einen anderes Ensemble beproben will muss man sich überlegen wie man dies am besten macht.
\end{notebox}
\begin{sectionbox}[\subsubsection{Replica Exchange Simulation/Parallel tempering}]\nospacing
  \begin{flalign*}
    &\imp{\text{Recall}:}&\nalign
    &&\hspace{-1cm}\Zz_{\text{klassisch}}(\Nz,\Vz,\Tz)\sim\Qzk=\int_{-\infty}^{\infty}\e^{-\betac\Wpot(\rvec^N)}\diff^{3\Nz}r
  \end{flalign*}
  \imp{Problem}: MD-Simulationen erfordern häufig \rd{ergodisches} Peproben von (Energie)konfigurationsräume mit vielen Minima und Barrieren
  zwischen Minima. Diese sind bei Raumtemperatur nur schwer zu durchequeren $\Rightarrow$ Die Ergebnisse der MD-Simulationen sind oft durch die Wahl der Startbedingunen (inital conditions) beschränkt.\\
  \imp{Frage}: können wir dieses Problem der limitierten Beprobung beheben?\\
  \imp{Idee}: Es geht uns draum das Integral zu sampeln, wie wir auf das Ergebnis kommen/die $\rvec_{\idxi}^N$ wählen ist nicht so relevant.\\
  $\Rightarrow$ simuliere simultan viele unabhängige Kopien von Ensembles/Systemen e.g. $\idxk$ und $\idxk'$ und vertausche die entsprechenden
  Konfigurationen $\rvec_{\idxk}^N$ mit $\rvec_{\idxk'}^N$ von Zeit zu Zeit.
  Typischerweise werden die Kopien so gewählt, dass das eine Extrem der Kopien das System ist das wir beproben wollen und das andere Extrem
  ein System ist in dem die Barrieren leichter überwunden werden können.\\
  Kopien können dabei simuliert werden für:
  \begin{numberlist}
    \item Verschiedene Thermodynamische Randbedingungen (z.B. Temperatur).
    \item Verschiedene Hamilton-Funktionen (z.B. verschiedene Kraftfelder).
  \end{numberlist}
\end{sectionbox}
\begin{sectionbox}[Vorgehen]\nospacing
  In bestimmten Intervallen werden nun die Konfigurationen/Koordinaten $\rvec_{\idxk}^N$ und $\rvec_{\idxk'}^N$ zweier Kopien $\idxk$ und
  $\idxk'$ mit einer Wahrscheinlichkeit $\pb$ ausgetauscht.
  \begin{align}
    \pb(\idxk\leftrightarrow\idxk')=\min\left(1,\e^{\mcc[\Delta]}\right):=\begin{cases}
        1 &\text{für } \mcc[\Delta]\leq0\\
        \e^{\mcc[\Delta]} &\text{für } \mcc[\Delta]>0
      \end{cases}
  \end{align}
  \begin{flalign*}
    &\rd{\text{Temperatur}}:&\mcc[\Delta]_{\Tz}\equiv\left[\frac{1}{\kb\Tz_{\idxk}}-\frac{1}{\kb\Tz_{\idxk'}}\right]\left[\Hamkl(\rvec_{\idxk'}^N)-\Hamkl(\rvec_{\idxk}^N)\right]
  \end{flalign*}
  \begin{flalign*}
    &\rd{\text{Verschiedene Kraftfelder}}:\mcc[\Delta]_{\Wpot}\equiv\nalign
    &\frac{\left[\Hamkl(\rvec_{\idxk'}^N,\Wpot_{\idxk})-\Hamkl(\rvec_{\idxk}^N,\Wpot_{\idxk})\right]-\left[\Hamkl(\rvec_{\idxk'}^N,\Wpot_{\idxk'})-\Hamkl(\rvec_{\idxk}^N,\Wpot_{\idxk'})\right]}{\kb\Tz}
  \end{flalign*}
\end{sectionbox}
%%% Local Variables:
%%% mode: latex
%%% TeX-master: "../formularySPCS"
%%% End: