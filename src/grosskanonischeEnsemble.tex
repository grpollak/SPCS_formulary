\subsection{Das Grosskanonische Ensemble $(\muz,\Vz,\Tz)$}
\label{subsec:DasGrosskanonischeEnsemble}
\begin{princpbox}
  \begin{princip}[Maximum Entropie Principle]
    Die Wahrscheinlichkeitsverteilung, die den momentanen Zustand am
    besten beschreibt ist diejenige, mit maximaler Entropie.
  \end{princip}
\end{princpbox}
\begin{sectionbox}[Problem]\nospacing
  \begin{wrapfigure}{r}{0.55\linewidth}	
    \centering{
      \inkscape[120pt]{figures/statPhysik/grosskanonischesEnsemble.pdf_tex}
    }
  \end{wrapfigure}
 Für verschiedene Experimente ist es Vorteilhaft anstelle Teilchenzahl
 $\Nz$, dass chemische Potential $\muz$ vorzugeben.\\
 \imp{Frage}: wie finden wir nun einen Ausdruck für die Anzahl an Mikrozuständen?\\
 \imp{Idee}: verallgemeinere das kanonische Ensemble durch Betrachtung
 vieler kanonischer Ensemble zu varierender Teilchenzahl $\Nz$ im System und Nutze das \rd{maximum Entropy Principle} um die
 Wahrscheinlichkeitsfunktion zu bestimmen.
 \begin{equation*}
   \Spots\eqs{\cref{defn:allgemeineEntropie}}-\kb\sum_{\idxr}\rd{\sum_{\Nz}}\left(\pb_{\idxr,\rd{N}}\ln\pb_{\idxr,\rd{N}}\right)
 \end{equation*}
 $\Rightarrow$ bestimmer $\{\pbrN\}$ so, dass $\Spots$ maximiert wird!\\
 \imp{Problem}: Nebenbedingungen für das grosskanonische Ensemble müssen eingehalten werden.
\end{sectionbox}
\begin{sectionbox}[Nebenbedingungen]\nospacing
  \begin{align*}
    &\sum_{N}\sum_{\idxr}\pbrN\eqs{!}1\nalign
    &\sum_{N}\sum_{\idxr}\pbrN\cdot\Ezr^{(N)}\eqs{!}\obs{\Ez}&(\Tz \text{ ist konstant})\nalign
    &\sum_{N}\sum_{\idxr}\pbrN\cdot N\eqs{!}\obs{\Nz}&(\muz \text{ ist konstant})
  \end{align*}
\end{sectionbox}
\todo[inline]{Add theorie of LMM somewhere}
\begin{sectionbox}[Lagrange Multiplikator Methode]\nospacing
  \begin{align*}
    \LT\equiv\Spot-&\alphac\left(\sum_{\idxr,N}\pbrN-1\right)-\betac\left(\sum_{\idxr,N}\left[\pbrN\Ezr^{(N)}\right]-\obs{\Ez}\right)\nalign
                     -&\gammac \left(\sum_{\idxr,N}\left[\pbrN N\right]-\obs{\Nz}\right)\qquad\text{\imp{Ziel: }} \delta\LT\eqs{!}0
  \end{align*}
  \text{Mit} $(\idxi,\idxM)\in\{(\idxr,N)\}$ und $\Ez_{\idxr}^{(\Nz)}:=\Ezr(\Vz,\Nz_{\idxr})$ folgt dann:
  \begin{alignat*}{3}
    \pfrac{\LT}{\pb_{\idxi,\idxM}}&\ltda{=}{\cref{eq:xlnx}}&&-\kb\overbrace{\pfrac{\pb_{\idxi,\idxM}\ln\pb_{\idxi,\idxM}}{\pb_{\idxi,\idxM}}}^{\ln\pb_{\idxi,\idxM}+1}
    -\alphac-\betac\Ez_{\idxi}^{(\idxM)}-\gammac\idxM\eqs{!}0
    \nalign
    &=&&-\kb\ln\pb_{\idxi,\idxM}-\kb-\alphac-\betac\Ez_{\idxi}^{(\idxM)}-\gammac\idxM\eqs{!}0
    \nalign
    \Rightarrow0&=&&-\ln\pb_{\idxi,\idxM}-\underbrace{\left(1-\frac{\alphac}{\kb}\right)}_{:=\alphac'}-\underbrace{\left(\frac{\betac}{\kb}\right)}_{:=\betac'}\Ez_{\idxi}^{\idxM}-\underbrace{\left(\frac{\gammac}{\kb}\right)}_{:=\gammac'}\idxM
    \nalign
    \Rightarrow\pb_{\idxi,\idxM}&=&&\e^{-\alphac'-\betac'\Ez_{\idxi}^{(\idxM)}-\gammac'\idxM}
  \end{alignat*} 
  \imp{Frage}: Wie können wir diesen Ausdruck nun weiter vereinfachen?\\
  \imp{Idee}: Benutze die Normierungsbedingung.
  \begin{alignat}{3}
    &&\sum_{N}\sum_{\idxr}\pbrN&=\ul{1}=\ul{\sum_{\idxr,N}e^{-\alphac'-\betac'\Ez_{\idxr}^{(N)}-\gammac'N}}\nonumber\nalign
    \Rightarrow&&\pb_{\idxi,\idxM}&=\frac{\pb_{\idxi,\idxM}}{\ul{1}}=\frac{\cancel{\e^{-\alphac'}}\e^{-\betac'\Ez_{\idxi}^{(\idxM)}}\e^{-\gammac'\idxM}}{\sum_{\idxr,N}\cancel{e^{-\alphac'}}\uldotted{\e^{-\betac'\Ez_{\idxr}^{(N)}}}\e^{-\gammac'N}}\nonumber\nalign
    &&&\eqs{\text{def.}}\frac{\e^{-\betac'\Ez_{\idxi}^{(\idxM)}}\e^{-\gammac'\idxM}}{\Zzg}\label{eq:grosskanonischeWahrscheinlichkeit}
  \end{alignat}
  Analog zum kanonischen Ensemble findet man $\betac'=\frac{1}{\kb\Tz}$ somit folgt:
  \begin{align*}
    \Zzg=\sum_{N}\underbrace{\uldotted{\Zzk}}_{\strut\mathclap{\text{kanonische Zustandsumme}}}\e^{-\gammac' N}
  \end{align*}
\end{sectionbox}
\begin{sectionbox}[Bedeutung von $\gammac$]\nospacing
  \begin{notebox}[Bemerkung]\nospacing
    Siehe auch Herleitung \Cref{subsec:BezugDruck}.
  \end{notebox}
  \begin{align}
    \diff\Upotp&=\ul[ulc2]{\Tz\diff\Spotp}-\ul{\pzp\diff\Vz}+\ul[ulc2]{\muzp\diff\Nz}\nalign
    \diff\Upots&=\diff\left(\sum_{\idxr,N}\pbrN\Ezr^{(N)}\right)\nalign
    &=\sum_{\idxr,N}\Ezr^{(N)}\diff\pbrN+\sum_{\idxr,N}\pbrN\diff\EzrN\nalign
    &=\ul[ulc2]{\sum_{\idxr,N}\Ezr^{(N)}\diff\pbrN}+\textcolor{ulc1}{\underbrace{\color{black}\sum_{\idxr,N}\pbrN
      \pfrac{\EzrN}{\Vz}\diff\Vz}_{\color{black}-\obs{\pz}\equiv-\pzs}}\nalign
    \Rightarrow&\sum_{\idxr,N}\Ezr^{(N)}\diff\pbrN=\ul[ulc4]{\Tz\diff\Spotp}+\ul[ulc3]{\muzp\diff\Nz}\label{eq:gammaGrosskanonisch}
  \end{align}
  Mit der Hilfe von \cref{eq:grosskanonischeWahrscheinlichkeit} erhalten wir einen Ausdruck für $\EzrN$:
  \begin{align*}
    &\pbrN=\frac{1}{\Zzg}\exp \left[-\frac{\EzrN}{\kb\Tz}-\gammac'N\right]\nalign
      \Rightarrow&\EzrN=-\kb\Tz\left\{\ln\pbrN+\ln\Zzg+\gammac'N\right\}
  \end{align*}
  Einsetzen in \cref{eq:gammaGrosskanonisch} liefert:
  \begin{align*}
    \sum_{\idxr,N}\Ezr^{(N)}\diff\pbrN&=-\kb\Tz\left\{\sum_{\idxr,N}\left[(\ln\pbrN)\diff\pbrN\right]\right.\nalign
  &\hspace{-4em}\left.+\ln\Zzg\overbrace{\sum_{\idxr,N}\diff\pbrN}^{\mathclap{=0,\text{da}\sum\pbrN=1}}+\gammac'\underbrace{\sum_{\idxr,N}N\diff\pbrN}_{\mathclap{\diff\obs{N}\equiv\diff N}}\right\}\nalign
    &\hspace{-4em}\eqs{\substack{\cref{eq:dprTrick}\\ \cref{eq:allgemeineEntropie}}}\ul[ulc4]{-\kb\Tz\rd{\diff}\left(\sum_{\idxr,N}\pbrN\ln\pbrN\right)}-\ul[ulc3]{\kb\Tz\gammac'\diff\Nz}
  \end{align*}
\end{sectionbox}
\begin{sectionbox}
\begin{align*}
    \Rightarrow   \gammac'=-\frac{\muz}{\kb\Tz}\text{ \imp{und} }
      \diff\Spots=-\kb\Tz\diff \left(\sum_{\idxr,N}\pbrN\ln\pbrN\right)
  \end{align*}
\end{sectionbox}
\begin{emphbox}\nospacing
  \begin{law}[Grosskanonische Zustandssumme]\leavevmode\\
    \begin{align}
      \Zzg&=\sum_{\idxj}\e^{-\betac \left(\Ez_{\idxj}(\Vz,\Nz_{\idxj})-\muz\Nz_{\idxj}\right)}\nonumber\nalign
      &=\sum_{\idxj}\Zzk\exp\left\{\frac{\muz}{\kb\Tz}N_{\idxj}\right\}\label{eq:Zzg}
    \end{align}
    \begin{align}
      &\text{Mit: }&\pb_{\idxr}=\frac{1}{\Zzg}\e^{-\betac\left(\Ez_{\idxr}^{(\Nz)}-\muz\Nz_{\idxr}\right)}\label{eq:probGrosskanonisch}
    \end{align}
  \end{law}
\end{emphbox}
	\vfill\columnbreak
\begin{sectionbox}[\subsubsection{Bezug zur Entropie}]\nospacing
  \begin{align*}
    &\text{Mit:}&\Spots\eqs{\text{def.\ref{defn:allgemeineEntropie}}}-\frac{1}{\betac\Tz}\sum_{\idxr}\left(\pbr\ln\pbr\right)
  \end{align*}
  und \cref{eq:Zzg} sowie \cref{eq:probGrosskanonisch} folgt:
\end{sectionbox}
\begin{sectionbox}\nospacing
  \begin{align*}
    \Spot^{\text{st.}}=&-\kb\sum_{\idxr}\frac{1}{\Zz}\e^{-\betac\left(\Ez_{\idxr}^{(\Nz)}-\muz\Nz_{\idxr}\right)}\ln \left[\frac{1}{\Zz}\e^{-\betac\left(\Ez_{\idxr}^{(\Nz)}-\muz\Nz_{\idxr}\right)}\right]
  \end{align*}
  \begin{align*}
    =&-\kb\frac{1}{\Zz}\sum_{\idxr}\e^{-\betac\left(\Ez_{\idxr}^{(\Nz)}-\muz\Nz_{\idxr}\right)}\left[-\ln\Zz-\betac\left(\Ez_{\idxr}^{(\Nz)}-\muz\Nz_{\idxr}\right)\right]\nalign
       =&\kb\frac{1}{\Zz}\underbrace{\sum_{\idxr}\e^{-\betac\left(\Ez_{\idxr}^{(\Nz)}-\muz\Nz_{\idxr}\right)}\ln\Zz}_{=\Zzg}\nalign
          &+\kb\betac\underbrace{\sum_{\idxr}\frac{1}{\Zz}\e^{-\betac\left(\Ez_{\idxr}^{(\Nz)}-\muz\Nz_{\idxr}\right)}\Ezr}_{\eqs{\cref{eq:ensembelDurchschnitt}}\obs{E}}\nalign
          &-\kb\betac\muz\underbrace{\sum_{\idxr}\frac{1}{\Zz}\e^{-\betac\left(\Ez_{\idxr}^{(\Nz)}-\muz\Nz_{\idxr}\right)}\Nz_{\idxr}}_{\eqs{\cref{eq:ensembelDurchschnitt}}\obs{N}}
  \end{align*}
\end{sectionbox}
\begin{defnbox}\nospacing
  \begin{defn}[\\Statistische Entropie \tc{black}{(gross kanonisch)}]\label{defn:EntropieKanonisch}
    \begin{equation}
      \Spots=\bdra{\kb}{\normalfont{Proportionalitätskonstante}}\ln\Zzm+\frac{\obs{\Ez}-\muz\obs{\Nz}}{\Tz}
    \end{equation}
  \end{defn}
\end{defnbox}
%%% Local Variables:
%%% mode: latex
%%% TeX-master: "../formularySPCS"
%%% End:
